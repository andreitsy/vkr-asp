% ОБЯЗАТЕЛЬНО ИМЕННО ТАКОЙ documentclass!
% (Основной кегль = 14pt, поэтому необходим extsizes)
% Формат, разумеется, А4
% article потому что стандарт не подразумевает разделов
% Глава = section, Параграф = subsection
% (понятия "глава" и "параграф" из стандарта)
\documentclass[a4paper,article,14pt]{extarticle}

% Подключаем главный пакет со всем необходимым
\usepackage{spbudiploma}

% Пакеты по желанию (самые распространенные)
% Хитрые мат. символы

\usepackage{euscript}
% Таблицы
\usepackage{longtable}
\usepackage{makecell}
% Картинки
\usepackage[pdftex]{graphicx}

\usepackage{amsthm, amssymb, amsmath, amsfonts, physics}
\usepackage{mathtext,cite,enumerate,float}
\usepackage{textcomp}
\usepackage[ruled,vlined]{algorithm2e}

\DeclareMathOperator{\Rot}{rot}
\DeclareMathOperator{\Div}{div}
\DeclareMathOperator{\Sp}{Sp}
\begin{document}

% Титульник в файле titlepage.tex
\begin{titlepage}
\newpage

\begin{center}
САНКТ-ПЕТЕРБУРГСКИЙ ГОСУДАРСТВЕННЫЙ УНИВЕРСИТЕТ\\
\vspace{1cm}

%\hrulefill
\end{center}

% \begin{flushright}
% На правах рукописи
% \end{flushright}
% \begin{flushright}
% \includegraphics[width=0.27\linewidth]{pic/signature.pdf}
% \end{flushright}

\vspace{0.5cm}
\begin{center}
ЦЫПИЛЬНИКОВ Андрей Васильевич
\end{center}

\vspace{1cm}
\begin{center}
    \textbf{выпускная квалификационная работа}
\end{center}{}
\vspace{1cm}

\begin{center}
\Large{\bf Спиновые волны в скирмионном кристалле}
\end{center}
\vspace{1cm}
\begin{center}
Направление 03.06.01 «Физика и астрономия» \\
Основная образовательная программа MK.3008.2016 «Физика»
\end{center}
\vspace{1cm}


\begin{flushleft}
\hspace{\stretch{1}} Научный руководитель:\\
\hspace{\stretch{1}} Аристов Дмитрий Николаевич\\
\hspace{\stretch{1}} д. ф.-м. н., проф.\\
\hspace{\stretch{1}} Рецензент:\\
\hspace{\stretch{1}} Демидов Юрий Андреевич\\
\hspace{\stretch{1}} к. ф.-м. н.\\
\vspace{1.5em}
\end{flushleft}

\vspace{\fill}

\begin{center}
Санкт-Петербург -- 2020
\end{center}
\end{titlepage}


% Содержание
\tableofcontents
\pagebreak

\specialsection{Введение}

Изучение различных существенно нелинейных явлений и эффектов в физике вызывает большой интерес как со стороны теоретиков, так и со стороны экспериментаторов.  Особую роль в анализе подобных нелинейных явлений играет топология --- раздел математики, изучающий свойства математических объектов, сохраняющиеся при непрерывных преобразованиях. Топология находит своё применение в различных областях физики и почти все современные теории содержать в себе те или иные её применения. Как пример, стоит упомянуть разделение солитонов в моделях квантовой теории поля \cite{rajaraman} с помощью топологических индексов или классификацию топологических изоляторов. \cite{kitaev} В данной работе мы интересуемся нелинейной $\sigma$-моделью, в рамках которой применяется классификация полевых конфигураций с помощью гомотопических индексов отображений сферы на сферу. Солитоны данной модели, впервые рассмотренные Т. Скирмом в моделях квантовой теории поля физики высоких энергий \cite{skyrme}, также обнаруживаются в теории магнитоупорядоченных сред. Традиционно данные солитоны называют скирмионами, но некоторые источники \cite{bogdanov} именуют их магнитными вихрями. Существование стабильных скирмионов было предсказано в тонких слоях магнитных материалов с нарушением инверсии, например в соединениях структурного типа $B20$.\cite{roslerBogdanov, bogdanov}


Свойства магнитоупорядоченных кристаллов в областях низких температур определяются спиновыми волнами, \cite{ahiezer} могущими распространяться в них. Согласно принципу квантово-волнового дуализма, для этих волн существуют связанные ними частицы – магноны. Сами по себе они во многом аналогичны фононам --- квантам звука в обычных твердых телах, однако обладают ещё и ненулевым магнитным моментом. Теория магнонов в простейшем типе магнитоупорядоченных кристаллов - ферромагнетиках – изучена достаточно хорошо. Подобная теория в магнетиках со скирмионным основным состоянием представляет большой интерес, особенно в связи с недавними экспериментальными работами, \cite{mulhbauer, yu} поскольку позволила бы лучше понять свойства скирмионных кристаллов. По большей части данная работа посвящена этой теме.


\begin{figure}[h]
	\centering
	\includegraphics[width=0.5\paperwidth]{images/skyrmionPic.png}
	\caption{Скирмион на гексагональной ячейке}
	\label{pic:skyrmion}
\end{figure}

Скирмионы наблюдаются в экспериментах как в виде одиночных возбуждений, стабильных за счёт неоднородности в кристалле\cite{rosch_muller}, так и в виде упорядоченной гексагональной решетки в $A$-фазе. \cite{mulhbauer} При этом 3D скирмионы могут рассматриваться как «стопки» из двумерных гексагональных решёток.\cite{milde, Rybakov2015}
На сегодняшний день удалось найти немало материалов, в которых обнаружили скирмионную решетку: металлы $\mathrm{Mn\,Si}$ и $\mathrm{Fe\,Ge}$, полупроводник $\mathrm{Fe_{1-x}\,Co_x\,Si}$ \cite{munzer} или изолятор $\mathrm{Cu_2\,O\, Se\, O_3}$. Эти вещества обладают различными электронными характеристиками, но одинаковыми магнитными свойствами. \cite{nagaosa}


\specialsection{Постановка задачи}
Целью данной работы является вывод уравнения Шрёдингера на спектр спиновых волн двумерного скирмионного кристалла и численное решение полученного уравнения. В работе подробно описывается предложенный полуаналитический метод для получения спектра и приводятся основные соотношения, необходимые для проведения численного анализа. Для получения матрицы используется базис функций Ландау, являющихся собственными функциями квантового уравнения движения на частицу в однородном магнитном поле в полярной системе координат. В рамках приведенного метода предлагается способ получения дисперсии магнонов для данной системы.

\specialsection{Обзор литературы}

Одной из первых работ, где были рассмотрены магнитные скирмионы, можно считать работу Белавина и Полякова. \cite{belavin} В этой работе была проанализирована стабильность конфигурации скирмиона на диске в классическом случае и показано, что решением уравнения Эйлера для простой обменной модели двумерного магнетика, может являться любая аналитическая функция. Было продемонстрировано, что состояние является метастабильным. В более поздних работах было замечено, что нарушение инверсии в магнитной системе приводит к стабилизации решения.\cite{bogdanov} Здесь стоит отметить, что большинство теоретических работ на эту тему ограничивались лишь классическим рассмотрением динамики скирмионов.\cite{kiselevBogdanov} Однако позже, в работах  \cite{aristov1, garst} был применен метод квазиклассического квантования и проведен анализ спектра магнонов в такой конфигурации. При этом исследовалась конфигурация с одним скирмионом и не рассматривался случай решетки. Анализ дисперсии магнонов на скирмионном кристалле был выполнен в работах \cite{garst_2017, roldan}, где также было показано, что различные состояния магнонов на сверхрешетке скирмионов обладают нетривиальными числами Черна. Важно отметить недавнюю работу \cite{back}, в которой отражены многие актуальные проблемы и ожидания исследователей в физике скирмионов.

Скирмионы интересны не только с точки зрения теории, но и с точки зрения различных технических приложений, например, с ними связывают надежды на создание сверхплотных, долговременных носителей информации \cite{fert} или транзисторов \cite{zhang}. Ещё интересны примеры взаимодействие скирмионов с другими квазичастицами вещества, например, они оказывают сильное воздействие на спиновый ток. Электрон, движущийся сквозь скирмион, несколько раз меняет спиновую ориентацию, подстраивая ее под локальное распределение намагниченности, в результате чего на него действует эффективная сила, изменяющая направление его движения. Макроскопически это должно проявляться как разновидность эффекта Холла.\cite{rosch_pfleiderer}


\pagebreak
\section{Вывод основного гамильтониана для одиночного изолированного скирмиона и построение классического решения}
\subsection{Базовый гамильтониан для двумерного хирального магнетика}

Магнетики могут быть рассмотрены в рамках макроскопической теории обменной модели Гейзенбeрга, которая вытекает из соответствующей микроскопической теории. Микроскопическая теория, в свою очередь, вытекает из законов релятивисткой квантовой теории.  Строгий вывод осуществляется путем рассмотрения квантово-механического принципа неразличимости тождественных частиц. В этом выводе энергия кристалла есть сумма кулоновских энергий отдельных атомов, образующих периодическую решетку. На волновую функцию кристалла накладываются требования принципа Ферми, что ведет к тому, что полная энергия начинает зависит от моментов импульса отдельных атомов. Вследствие этого, конечная энергия, с определенной долей точности, может быть получена добавлением к исходному гамильтониану эффективной энергии, зависящей от моментов импульса $\widehat{S}_i^\alpha$ отдельных атомов на решетке. Латинский индекс, здесь и далее по тексту, отвечает пространственным переменным, а греческий – спиновым. Подразумеваться суммирование по повторяющимся индексам. Будем считать, что $\widehat {\mathbf{S}}_{i}$ обозначает вектор в спиновом пространстве. Операторы $\widehat{S}$ удовлетворяют коммутационным соотношениям $[\widehat{S}_i^\alpha, \widehat{S}_j^\beta]=i \delta_{ij} \epsilon_{\alpha \beta \gamma} \widehat{S}_i^\gamma $. Используя эти операторы и обменные интегралы $J(\mathbf{r}_i - \mathbf{r}_j)$, мы можем записать энергию для двух атомов как $J\left(\mathbf{r}_i - \mathbf{r}_j \right){{\widehat {\mathbf{S}}}_i}{{\widehat {\mathbf{S}}}_j}$, при этом  функция $ J$ такая, что $J\left(\mathbf{r}_i - \mathbf{r}_j \right)<0$ и $J(\mathbf{r})=J(r)\simeq -e^{-r}$.

Будем учитывать, в дополнение к обменному слагаемому, и другие вклады в энергию. Поскольку мы рассматриваем двумерную магнитную систему без центра инверсии, то мы включаем в рассмотрение взаимодействие Дзялошинского-Мория, которое может быть представлено для пары спинов как 
$${H_{DM}} = {{\mathbf{D}}_{ij}} \cdot \left( {\widehat {\mathbf{S}}}_{i} \times {{\widehat {\mathbf{S}}}_j} \right), $$
с вектором $\mathbf{D}_{ij} = D \left(\mathbf{r}_i - \mathbf{r}_j \right)$, лежащим в плоскости. \cite{lacrox} Данное взаимодействие является одним из проявлений спин-орбитального взаимодействия, которое, будучи релятивистским эффектом, вносит существенный вклад в энергию в ряде кристаллов.\cite{Luo2019} Учтем ещё Зеемановский вклад, считая, что в системе есть внешнее магнитное поле $\mathbf{H}_e$ направленное перпендикулярно к плоскости. В некоторых работах дополнительно учитывают еще анизотропию типа «легкая ось», \cite{bogdanov} но мы не будем включать её в рассмотрение, поскольку общая картина, после добавления этого слагаемого, существенно не изменяется. Как будет показано далее, подбирая параметры модели с учетом лишь упомянутых выше слагаемых, мы можем получить стабильную конфигурацию скирмионного кристалла. Полный гамильтониан микроскопической теории принимает следующий вид:


\begin{equation}
\label{eq:ham_init}	
H = \sum\limits_{i,j} {\left[ {J\left( {{{\mathbf{r}}_i} - {{\mathbf{r}}_j}} \right){{\widehat {\mathbf{S}}}_i}{{\widehat {\mathbf{S}}}_j} + {{\mathbf{D}}_{ij}}  {{\widehat {\mathbf{S}}}_i} \times {{\widehat {\mathbf{S}}}_j}} \right] - \mathbf{H}_e} \sum\limits_{i} {{{\widehat {\mathbf{S}}}_i}}
\end{equation}

Осуществим переход к макроскопической теории и, для этой цели, определим оператор плотности магнитного момента импульса
\begin{equation}
\label{eq:densSpinOperator}	
{\widehat {\mathbf{S}}} (\mathbf{r}) = \sum_i {{\widehat {\mathbf{S}}}_i} \delta (\mathbf{r} - \mathbf{r}_i)
\end{equation}
и соответствующий оператор макроскопической плотности магнитного момента импульса
\begin{equation}
\label{eq:densSpinVec}	
{\mathbf{S}}_\mathbf{r} \equiv {\mathbf{S}} (\mathbf{r}) = \frac{1}{v_0} \int\limits_{v_0} {\Sp{\left( {\widehat {\mathbf{S}}} (\mathbf{r'}) {\widehat \rho  } (\mathbf{r'}) \right)} d \mathbf{r'} }
\end{equation}
где под ${ \widehat \rho }$ мы понимаем матрицу плотности рассматриваемого вещества. Интегрирование в \eqref{eq:densSpinVec} производится в пределах физически малого объема $v_0$ с центром в точке $\mathbf{r}$. Будем считать при этом, что $a_0^3  \ll v_0 \ll \ell^3 $, где $a_0$ - межатомное расстояние, а $\ell$ - некоторый характерный размер неоднородности, в данном случае - это постоянная сверхрешетки скирмионов. Значение оператора спина $s$ предполагается большим для большинства реальных магнетиков. Что означает следующие условие $\widehat{\mathbf{S}}^{2}({\mathbf{r}}) = s(s+1) \gg 1$. В пределе $s\to \infty$, магнитный вектор становится неотличим от классического вектора намагниченности с длиной $\mu = s/v_{0}$.


Теперь, усредняя вектора \eqref{eq:ham_init}, мы заменяем произведение операторов моментов импульса произведением средних локальных плотностей (\ref{eq:densSpinVec}), являющихся $c$-числами, и получаем следующее выражение для энергии
\begin{equation}
\label{eq:ham}	
H = \int\limits_{\mathbf{r}, \mathbf{r}'} \left[ J\left( \mathbf{r} - \mathbf{r}' \right)\mathbf{S}_{\mathbf{r}} \mathbf{S}_{\mathbf{r}'} + {{\mathbf{D}}_{\mathbf{r} \mathbf{r}'}} \cdot \mathbf{S}_{\mathbf{r}} \times \mathbf{S}_{\mathbf{r}'} \right] -  \mathbf{H}_e  \int\limits_{\mathbf{r}}\mathbf{S}_{\mathbf{r}}
\end{equation}

\subsection{Описание сверхрешетки скирмионов}
Как уже было сказано выше, экспериментально наблюдается, что скирмионы в магнетике образуют гексагональную решетку в одной из плоскостей. \cite{mulhbauer, yu} Трехмерные скирмионные структуры можно представить как «многослойку» подобных двумерных решеток, наложенных друг на друга вдоль оси перпендикулярной плоскости с гексагональной решеткой. Мы будем рассматривать только двумерный случай, и использовать следующее приближение: вместо рассмотрения плоскости, замощенной правильными шестиугольниками, мы рассмотрим замощение плоскости дисками. Это означает, что соответствующая гексагональная ячейка аппроксимируется кругом равной площади $S_{sk}$ (Рис. \ref{pic:approxHexagone}). Такое приближение позволяет избежать сложностей, связанных с рассмотрением нетривиальной границы гексагональной ячейки, и использовать полярные координаты для описания конфигурации скирмиона. Мы полагаем, что различие между шестиугольником и диском на границе не должно сильно влиять на суммарную энергию, поскольку вектора локальной намагниченности являются практически коллинеарными и сонаправленными на краю скирмиона, и это предположение укладывается в рамки допустимых приближений в данной работе.

\begin{figure}[h]
\centering
\includegraphics[width=0.5\paperwidth]{images/approx.png}
\caption{Гексагональная ячейка аппроксимируется диском равной площади.}
\label{pic:approxHexagone}
\end{figure}

При этом, радиус $R$ диска выбирается таким, чтобы энергия образца с решеткой была минимальна. \cite{bogdanov, nagaosaHan} Будем обозначать диск оптимально радиуса как $R_0$. Площадь гексагональной ячейки должна быть равна этому диску. Длина стороны шестиугольника $\ell$ определяется из условия равенства площадей $\pi R_0^{2} = \sqrt{3}\,\ell^{2}/2$. Поскольку полная энергия решетки получается суммированием энергий отдельных скирмионов, то для минимизации её достаточно минимизировать плотность энергии одного скирмиона на диске.


\subsection{Вывод уравнения на спектр спиновых волн из решеточной модели}
Используя аналогичный работе \cite{aristov1} подход, получим уравнение на спектр магнонов в конфигурации с одним скирмионом в начале координат. Как упоминалось выше, введем тензорные обозначения с латинскими индексами для спиновых переменных ($a,b,c,d = 1,2,3$) и греческими для пространственных ($\alpha, \beta, \gamma = 1,2$).  Меняя порядок суммирования и вводя вектор $\mathbf{n}=\mathbf{r} - \mathbf{r}'$, получаем из \eqref{eq:ham} выражение в таком виде

\begin{equation}
\label{eq:mainHam}
\int\limits_{{\mathbf{r}},{\mathbf{n}}} {\left\{ { S_{\mathbf{r}}^a\left( {J\left( {\mathbf{n}} \right){\delta _{ab}} + D{\varepsilon _{abc}}{\delta _{\alpha c}}{n^\alpha }} \right) S_{{\mathbf{r}} - {\mathbf{n}}}^b} \right\}}  - s H_e \int\limits_{\mathbf{r}} { \delta _{a3}  S_{\mathbf{r}}^a }
\end{equation}
Здесь мы ввели антисимметричный тензор $\varepsilon_{abc}$ ($\varepsilon_{123} = 1$) и магнитное поле $\mathbf{H}_e$ направленное перпендикулярно плоскости. Для удобства мы измеряем величину $\mathbf{H}_e$ в единицах $s$. Заметим, что взаимодействие Дзялошинского-Мория смешивает спиновые и пространственные индексы - это известное свойство спин-орбитального взаимодействия.

Предположим теперь, что основное состояние --- это скирмион, и перепишем гамильтониан в таком локальном базисе, что средняя намагниченность будет направлена вдоль локальной оси $z$.

Переход к подобному базису ${\mathbf{S}}_{\mathbf{r}} = \hat U\left( \mathbf{r} \right)\tilde { \mathbf{S}}_{\mathbf{r}}$ осуществляется с помощью некоторой матрицы $\hat U(\mathbf{r}) = e^{-\alpha \sigma_3}e^{-\beta \sigma_2}e^{-\gamma \sigma_3}$, с соответствующими генераторами $\sigma_2$,$\sigma_3$ группы $SO(3)$ и углами Эйлера $\alpha$, $\beta$, $\gamma$. Угол $\gamma$ является произвольным, и мы не можем определить его из условий минимума энергии классической конфигурации. Однако он может быть выбран, исходя из соображений того, \cite{aristov1}, требуем ли мы однозначность и непрерывность матрицы $U$ в нуле ($\gamma = -\alpha$) или на границе ($\gamma = \alpha$). В дальнейшем мы положим  $\gamma = -\alpha$.

Проведем градиентное разложение оператора макроскопической плотности магнитного момента импульса. Будем считать, что $|\mathbf{n}|=n \ll |\mathbf{r}|$. Тогда можно воспользоваться разложением $${S}_{{\mathbf{r}} - {\mathbf{n}}}^b = {S}_{\mathbf{r}}^b - {n^\beta }{\nabla ^\beta }{S}_{\mathbf{r}}^b + \frac{1}{2}{n^\beta }{n^\gamma }{\nabla ^\beta }{\nabla ^\gamma }{S}_{\mathbf{r}}^b,$$ где слагаемые с градиентами имеют следующий вид в новом базисе:

\[{\nabla ^\beta }S_{\mathbf{r}}^b = {\nabla ^\beta }\left( {{U^{bd}}{{\tilde S}^d}_{\mathbf{r}}} \right) = \left( {{\nabla ^\beta }{U^{bd}}} \right){{\tilde S}^d}_{\mathbf{r}} + {U^{bd}}\left( {{\nabla ^\beta }{{\tilde S}^d}_{\mathbf{r}}} \right)\]

	
\[{\nabla ^\beta }{\nabla ^\gamma }S_{\mathbf{r}}^b = \left( {{\nabla ^\beta }{\nabla ^\gamma }{U^{bd}}} \right){\tilde S^d}_{\mathbf{r}} + 2\left( {{\nabla ^\beta }{U^{bd}}} \right)\left( {{\nabla ^\gamma }\tilde S_{\mathbf{r}}^d} \right) + {U^{bd}}\left( {{\nabla ^\beta }{\nabla ^\gamma }{{\tilde S}^d}_{\mathbf{r}}} \right)\]
Удобно определить два тензора

\begin{eqnarray}
\label{eq:chi}
\begin{gathered}
  \chi _{1,\alpha }^{ab} = {U^{ca}}{\nabla ^\alpha }{U^{cb}}, \hfill \\
  \chi _{2,\alpha \beta }^{ab} = {U^{ca}}\left( {{\nabla ^\alpha }{\nabla ^\beta }{U^{cb}}} \right) \hfill \\ 
\end{gathered}
\end{eqnarray}
Явные выражениями для матриц $U(\mathbf{r})$ и $\chi_1(\mathbf{r})$, $\chi_2(\mathbf{r})$ известны \cite{paper:aristov} и мы не будем приводить их в данной работе.

В длинноволновом приближении (${\mathbf{qn}} \ll 1$) получаем, что
\[J({\mathbf{q}}) = \sum\limits_{\mathbf{n}} {{e^{i{\mathbf{qn}}}}J({\mathbf{n}})}  \simeq J(0) + \frac{C}{2}{q^2}\]
Если мы вычислим внутренние суммы по $\mathbf{n}$, то получим для обменного слагаемого выражение

\begin{eqnarray}
\label{eq:integrExch}
\begin{gathered}
  \sum\limits_{\mathbf{n}} {{n^\alpha }J({\mathbf{n}})}  = 0 \hfill \\
  \sum\limits_{\mathbf{n}} {{n^\alpha }{n^\beta }J({\mathbf{n}})}  =  - C{\left. {\frac{{{d^2}J({\mathbf{q}})}}{{d{q^\alpha }d{q^\beta }}}} \right|_{q = 0}} = - C{\delta _{\alpha \beta }} \hfill \\ 
\end{gathered}
\end{eqnarray}
и выражение для вклада от взаимодействия Дзялошинского-Мория:

\begin{eqnarray}
\label{eq:integrDM}
\begin{gathered}
  D\sum\limits_{\mathbf{n}} {{\varepsilon _{abc}}{\delta _{\alpha c}}{n^\alpha }}  = 0 \hfill \\
  D\sum\limits_{\mathbf{n}} {{\varepsilon _{abc}}{\delta _{\alpha c}}{n^\alpha }{n^\beta }}  = D{\varepsilon _{abc}}{\delta _{\beta c}} \hfill \\ 
\end{gathered}
\end{eqnarray}

Используя (\ref{eq:chi}),(\ref{eq:integrExch}) и (\ref{eq:integrDM}), получаем из (\ref{eq:mainHam}) гамильтониан в следующем виде
\begin{equation}
\label{eq:hamlitNewBasis}
H \approx {H_{ex}} + {H_{DM}} + {H_Z}
\end{equation}
где $H_{ex}$ - это слагаемое, вытекающее из обменного взаимодействия,
\[{H_{ex}} =  - \frac{1}{2}C\int\limits_{\mathbf{r}} {\tilde S_{\mathbf{r}}^a\left( {\chi _{2,\beta \beta }^{ab} + 2\chi _{1,\beta }^{ab}{\nabla ^\beta } + {\delta _{ab}} \nabla ^ 2 } \right)\tilde S_{\mathbf{r}}^b} \]
$H_{DM}$ слагаемое от Дзялошинского-Мория
\[{H_{DM}} =  - D\int\limits_{\mathbf{r}} {\tilde S_{\mathbf{r}}^a{\varepsilon _{adc}}{\delta _{e\alpha }}{U^{ec}}\left( {\chi _{1,\alpha }^{db} + {\delta _{db}}{\nabla ^\alpha }} \right)\tilde S_{\mathbf{r}}^b} \]
и, наконец, $H_Z$ - это Зеемановский вклад 
\[{H_Z} =  - s H_e \int\limits_{\mathbf{r}} {{U^{3a}}\tilde S_{\mathbf{r}}^a} \]
Воспользуемся представлением Малеева-Дайсона для спиновых операторов, сохраняющим коммутационные соотношения  ($[\tilde{S}^a,\tilde{S}^b] = i \epsilon_{abc}\tilde{S}^c$):
\begin{equation} 
\begin{aligned} 
\label{eq:boz}
     \tilde{S}^{z}_{j} &=s-a^+_{ j} a_{ j} \,, \quad   
       \tilde{S}^{+}_{j}=\sqrt{2s}a_{ j}  \\
     \tilde{S}^{-}_{j} &=\sqrt{2s}\left( a^{+}_{ j} - \frac{1}{2s}a^+_{ j}a^{+}_{ j}a_{ j} \right)
  \end{aligned}  
 \end{equation} 
здесь $s$ - это величина спина,  $\tilde S^{\pm} = \tilde S^{x} \pm i \tilde S^{y}$ и $[a_{ j},a^+_{ j}] = 1$.  
С помощью (\ref{eq:boz}), мы получаем из (\ref{eq:hamlitNewBasis}) энергию в ввиде ряда по степеням $s$, слагаемое при $s^2$ имеет вид
\begin{equation}
\label{eq:cls_skx_eng}
{H_c} = \int {d\mathbf{r} \left( {-J\left( 0 \right) + \frac{1}{2}{\text{C}}\left( {\frac{{{{\sin }^2}\beta }}{{2{r^2}}} + {{\left( {\frac{{d\beta }}{{dr}}} \right)}^2}} \right){\text{ + D}}\left( {\frac{{\sin 2\beta }}{{2r}} + \frac{{d\beta }}{{dr}}} \right) - H_e \cos \beta } \right)}
\end{equation} 
Используя уравнение Эйлера на $\beta$ для скирмионна \eqref{eq:eulerLagr}, можно убедиться, путем интегрирования по частям, что слагаемые при степени $s^{3/2}$ сокращаются --- это соответствует тому, что в качестве основного состояния взято скирмионное решение, отвечающее локальному минимуму полной энергии \eqref{eq:ham}. 

В порядке по $s$, получаем квадратичный по операторам рождения и уничтожения гамильтониан в $r$-представлении:

\begin{equation}
\label{eq:HamQuantBose}
{H_q} = \frac{1}{2}\int {d\mathbf{r} \left( {2a_{\mathbf{r}}^\dag \hat F\left( {\mathbf{r}} \right){a_{\mathbf{r}}} + a_{\mathbf{r}}^\dag G^*\left( {\mathbf{r}} \right)a_{\mathbf{r}}^\dag  + {a_{\mathbf{r}}}{G}\left( {\mathbf{r}} \right){a_{\mathbf{r}}}} \right)}
\end{equation}

\[\begin{gathered}
  F({L_z}) \equiv C\left( { - \nabla ^ 2  + \frac{{1 + 3\cos 2\beta }}{{4{r^2}}} - \frac{{2\cos \beta }}{{{r^2}}}{{\text{L}}_{\text{z}}} - \frac{1}{2}{{\left( {\frac{{d\beta }}{{dr}}} \right)}^2}} \right) \hfill \\
  \,\,\,\,\,\,\,\,\,\,\,\,\,\, + D\left( { - \frac{{3\sin 2\beta }}{{2r}} + \frac{{2\sin \beta }}{r}{{\text{L}}_{\text{z}}} - \frac{{d\beta }}{{dr}}} \right) + H_e \cos \beta  \hfill \\ 
\end{gathered} \]

\[G \equiv \frac{C}{2}\left( { - \frac{{{{\sin }^2}\beta }}{{{r^2}}} + {{\left( {\frac{{d\beta }}{{dr}}} \right)}^2}} \right) + {\text{D}}\left( {\frac{{d\beta }}{{dr}} - \frac{{\sin 2\beta }}{{2r}}} \right)\]
где определены операторы ${{\text{L}}_{\text{z}}} \equiv  - i\frac{\partial }{{\partial \varphi }}$
и 
$\nabla ^ 2  \equiv \frac{1}{r}\frac{\partial }{{\partial r}}\left( {r\frac{\partial }{{\partial r}}} \right) - \frac{{{\text{L}}_z^2}}{{{r^2}}}$ 

Это уравнение является главной целью первой главы. Оно выписано с сохранением всех констант взаимодействий для того, чтобы лучше проиллюстрировать какие слагаемые отвечают тому или иному взаимодействию, в дальнейшем  будет удобно использовать безразмерные единицы измерения.

\subsection{Единицы измерения и функционал плотности классической энергии скирмиона на диске }
Мы получили выражение для классической энергии скирмиона \eqref{eq:cls_skx_eng}, из которого следует исключить константу $-J(0)$ и ввести более удобные единицы измерения для данной задачи. В качестве единицы расстояния удобно взять отношение $C/D$, а для единицы энергии $E_{0} = \mu D^{2}/ C$. Тогда в \eqref{eq:cls_skx_eng} остается единственный безразмерный параметр $b$ равный $H_e C/(\mu D^2)$. Во всех дальнейших рассуждениях, связанных с численными расчётами, мы будем полагать $b=0.6$, если значение $b$ не оговорено особо.  Классический вклад в энергию имеет большой префактор в энергию $\mu \sim s$ и определяется статической конфигурацией $\mathbf{S}(\mathbf{r}) $, минимизирующей энергию $\langle H\rangle$.  Легко убедиться, что вклад от обменного взаимодействия в подынтегральном выражении в энергию принимает вид


\begin{equation}
\label{eq:ExchClass}
 {\left( {\nabla {\mathbf{S}}} \right)^2} = {\left( {\frac{{\partial \beta }}{{\partial r}}} \right)^2} + \frac{1}{{{r^2}}}{\left( {\frac{{\partial \beta }}{{\partial \varphi }}} \right)^2} + {\sin ^2}\beta \left( {{{\left( {\frac{{\partial \alpha }}{{\partial r}}} \right)}^2} + \frac{1}{{{r^2}}}{{\left( {\frac{{\partial \alpha }}{{\partial \varphi }}} \right)}^2}} \right),
\end{equation}
от Дзялошинского-Мория

\begin{equation}
\label{eq:DMclass}
\begin{gathered}
  {\mathbf{S}} \cdot \left[ {\nabla  \times {\mathbf{S}}} \right] = \left( {\frac{{\partial \alpha }}{{\partial r}}\cos \left( {\alpha  - \varphi } \right) + \frac{1}{r}\frac{{\partial \alpha }}{{\partial \varphi }}\sin \left( {\alpha  - \varphi } \right)} \right)\sin \beta \cos \beta  \hfill \\
  \,\,\,\,\,\,\,\,\,\,\,\,\,\,\,\,\,\,\,\,\,\,\, + \frac{{\partial \beta }}{{\partial r}}\sin \left( {\alpha  - \varphi } \right) - \frac{1}{r}\frac{{\partial \beta }}{{\partial \varphi }}\cos \left( {\alpha  - \varphi } \right) \hfill \\ 
\end{gathered},
\end{equation}
от Зеемеановской энергии

\begin{equation}
\label{eq:ExtClass}
{\mathbf{H}_e} \cdot {\mathbf{S}} =  b \cos \, \beta 
\end{equation}

Собирая вклады, мы можем получить общее выражение для классического гамильтониана в полярных координатах. 

Для того, чтобы получить уравнение Эйлера-Лагранжа на скирмионную конфигурацию, необходимо задать граничные условия на функции $\alpha$ и $\beta$. С этой целью необходимо обратиться к топологическим свойствам скирмионов, что и будет сделано в следующем разделе. Однако, сразу отметим, что мы будем интересоваться только скирмионом с $Q=-1$ и $\gamma_0=\pi/2$. Если мы захотим минимизировать плотность энергии скирмиона на диске с радиусом $R_0$, аналогично работе \cite{bogdanov}, то нам нужно будет записать энергию на диске в безразмерных единицах и поделить её на площадь диска. При этом удобно будет вычесть постоянную энергию основного состояния ферромагнетика $b$ (так, чтобы при $\beta \left( r \right) \equiv 0$ энергия была равна нулю). Соответствующий функционал плотности энергии принимает в таком случае  вид

\begin{equation}
\label{eq:density}
{\rho _c} = \frac{2}{{R_0^2}}\int\limits_0^{{R_0}} {dr\left( {\frac{{{{\sin }^2}\beta }}{r} + r{{\left( {\frac{{d\beta }}{{dr}}} \right)}^2} + r\frac{{d\beta }}{{dr}} + \frac{{\sin 2\beta }}{2} - br\left( {\cos \beta  - 1} \right)} \right)}
\end{equation}

\subsection{Топологический заряд}

Известно\cite{rajaraman}, что отображения вида $\mathcal{S}^2 \rightarrow \mathcal{S}^2$ могут быть разбиты на гомотопические секторы. Множество таких секторов или классов счётно и может быть классифицировано набором целых чисел $Q$. Это число, выделяющее ту или иную конфигурацию, называют топологическим зарядом. В нашем случае, $Q$ определяет число полных обходов образа отображения координатного пространства $\mathbb{R}^2 \bigcup \{\infty\}$, которое топологически эквивалентно $\mathcal{S}^2$ за счет условия нормировки.

Структура скирмиона определяется топологическим зарядом \cite{rajaraman}:

\begin{equation}
\label{eq:topChargeMain}
Q \equiv \frac{1}{{4\pi }}\int {d{\mathbf{r}}\left( {{\mathbf{S}}\left[ {\frac{{\partial {\mathbf{S}}}}{{\partial x}} \times \frac{{\partial {\mathbf{S}}}}{{\partial y}}} \right]} \right)}
\end{equation}
Используя параметризацию для единичного вектора локальной намагниченности $\mathbf{S}$:
\begin{equation}
\label{eq:parametr}
\mathbf{S} = \left( {\begin{array}{*{20}{c}}
{\cos \alpha \sin \beta }\\
{\sin \alpha \sin \beta }\\
{\cos \beta }
\end{array}} \right)
\end{equation}
и полагая $\alpha = \alpha(\phi)$ и $\beta = \beta (r)$, выражение \eqref{eq:topChargeMain} преобразуется к
\begin{equation}
\label{eq:topChargeSimplify}
\left. {Q = \frac{1}{{4\pi }}\int\limits_0^{{R_0}} {dr\int\limits_0^{2\pi } {d\varphi } } \frac{{d\alpha }}{{d\varphi }}\frac{{d\beta }}{{dr}}\sin \beta  =  - \frac{1}{{2\pi }}\alpha \left( \varphi  \right)} \right|_{\varphi  = 0}^{\varphi  = 2\pi }\left. {\frac{1}{2}\cos \beta \left( r \right)} \right|_{r = 0}^{r = {R_0}}
\end{equation}
Отсюда можно заключить, что для задания нетривиальной топологической конфигурации ($Q \neq0$), следует взять граничные условия вида

\begin{equation}
\label{eq:edgeCond}
{\left\{ {\begin{array}{*{20}{c}}
  {\alpha \left( \varphi  \right) = {W_0}\varphi  + {\gamma _0}} \\ 
  {\beta (0) = \pi } \\ 
  {\beta ({R_0}) = 0} 
\end{array}} \right.}
\end{equation}
Здесь $W_0$ это целое число, а $\gamma _0$ произвольная константа. При таких условиях, мы получаем $Q=-W_0$. Известно, что энергия скирмиона тем выше, чем больше $|Q|$ \cite{rajaraman, belavin}, поэтому будем рассматривать конфигурации с $Q=\pm 1$. В данной работе анализируется "baby skyrmion" с $Q=-1$, следовательно $W_0 = 1$.

В отсутствии стабилизирующих взаимодействий, энергия не зависит от $\gamma_0$. В нашем случае это не так. В самом деле, используя \eqref{eq:DMclass} и \eqref{eq:edgeCond} получаем

\[\mathbf{D} \, {\mathbf{S}} \cdot \left[ {\nabla  \times {\mathbf{S}}} \right] = D \, \sin \left( {\left( {{W_0} - 1} \right) \varphi  + {\gamma _0}} \right)\,\left( {\frac{{d\beta }}{{dr}} + {W_0}\frac{{\sin 2\beta }}{{2r}}} \right)\]
поскольку мы хотим минимизировать общую энергию, это слагаемое должно быть отрицательным, для этого и интеграл от него тоже должен быть отрицательным. Поскольку $\int {d{r}} \left( {\frac{{d\beta }}{{dr}} + {W_0}\frac{{\sin 2\beta }}{{2r}}} \right) < 0$ для скирмионной конфигурации  при $W_0=1$,  заключаем, что множитель перед интегралом должен быть положительным. Этот вклад в энергию будет максимальным при $\gamma_0 = \pi/2$ если $D>0$. В этом выражении восстановлена константа $D$, чтобы показать, что направление закрутки скирмиона $\gamma_0 = \pm \pi/2$ определяется направлением вектора Дзялошинского-Мории.


\subsection{Асимптотики профиля скирмиона $\beta$ и плотность классической энергии }

Уравнение Эйлера-Лагранжа для \eqref{eq:density} с граничными условиями \eqref{eq:edgeCond} на $\beta$ принимает вид
\begin{equation}
\label{eq:eulerLagr}
\frac{{{d^2}\beta }}{{d{r^2}}} + \frac{1}{r}\frac{{d\beta }}{{dr}} - \frac{{\sin \beta \cos \beta }}{{{r^2}}} + \frac{{2{{\sin }^2}\beta }}{r} - b\sin \beta  = 0
\end{equation}
Данное уравнение выглядит довольно громоздко и решить его аналитически не представляется возможным. Однако оно допускает несложный численный анализ и позволяет получить профиль скирмиона путем интерполяции численного решения для некоторого радиуса диска.\cite{bogdanov} Подставляя профиль $\beta$ в (\ref{eq:density}), и вычисляя $\rho_c$, получаем зависимость плотности энергии от радиуса диска, зависимость которой от радиуса приведена на Рис. \ref{pic:plotDensity}.

Асимптотики решения уравнения \eqref{eq:eulerLagr} имеют вид:
\begin{eqnarray*}
\label{eq:asympt_beta}
\beta(r) &\sim \pi - c_1 \sqrt{b} r ,\qquad &r \rightarrow 0 \\
\beta(r) &\sim \frac{c_2}{b^{\frac{1}{4}}\sqrt{r}} e^{-\sqrt{b} r} ,\qquad & r \rightarrow \infty
\end{eqnarray*}
с некоторыми коэффициентами $c_1>0$ и $c_2>0$. Из асимптотики $\beta$ можно заключить, что величина $\frac{1}{\sqrt{b}}$ определяет характерный радиус скирмиона.

\begin{figure}[h]
\centering\includegraphics[width=0.65\paperwidth]{images/plotDensity.pdf}
\caption{Зависимость $\rho _c$ от радиуса диска при различных магнитных полях (параметр $b$). При $b\gtrsim 0.8$ не наблюдается минимум и энергия скирмиона положительна.}
\label{pic:plotDensity}
\end{figure}
Из (рис. \ref{pic:plotDensity}) видно, что существует оптимальный радиус $R_0$ минимизирующий плотность в определенном интервале величин внешнего магнитного поля. При достаточно большом поле (при $b > b_{crit} \approx 0.8$) не наблюдается минимум энергии скирмиона и скирмионное состояние метастабильно.

\subsection{ Сравнение плотностей энергий скирмионной решетки и спирали }

Стоит отметить, что одиночная спираль тоже является хорошим кандидатом для истинного основного состояния системы описываемой уравнением \eqref{eq:density}. Плотность энергии для конической спирали в 3D случае без ферромагнитного вклада $-B$, дается выражением \cite{Maleyev2006}
 \begin{equation}
\begin{aligned}
\rho_{con} & = - \frac{D^{2}}{2C} (b-1)^{2} \,, \quad  0<b<1   \,, \\
 &= 0   \,, \quad  b \ge1 \,,  \\
\end{aligned}
\label{helix_energy}
\end{equation}
В 3D случае ось конуса конической спирали $q=D/C$ параллельна направлению поля, в 2D мы имеем следующее выражение для плотности энергии
 \begin{equation}
\begin{aligned}
\rho_{hel} & = \frac{D^{2}}{C} (-1/2+b) \,, \quad  0<b<1/2   \,, \\
 &= 0   \,, \quad  b \ge1/2 \,,  \\
\end{aligned}
\label{helix_energy2D}
\end{equation}
Для небольших полей ось конуса лежит в плоскости и его апертура равна $ \pi $ (геликоида); для более сильных полей ($ b > 1/2 $) спираль имеет большую энергию, чем конфигурация с равномерным спином.

На Рис.\ref{fig:Sk_v_helix} мы сравниваем плотности энергий описываемых уравнениями \eqref{helix_energy}, \eqref{helix_energy2D} с плотностью энергии определяемой скирмионной конфигурацией уравнения \eqref{eq:density}.


\begin{figure}[t]
\centering	
\includegraphics[width=0.85\columnwidth]{images/Sk_v_helix.pdf}
\caption{Классическая плотность энергии скирмиона (пунктирная линия, уравнение \eqref{eq:density}) показана в сравнении с энергией конической спирали в 3D (сплошная линия, уравнение \eqref{helix_energy}) и геликоидой в 2D (пунктирная линия, уравнение \eqref{helix_energy2D}). Кривые нормированы множителем $s$.}
\label{fig:Sk_v_helix}
\end{figure}

Отсюда видно, что в 2D случае скирмионная конфигурация выигрывает по энергии для полей  $0.17 \lesssim b \lesssim 0.8$, в соответствии с \cite{Rybakov2015}.
Энергия конической спирали имеет меньшее значение, чем у скирмионной конфигурации во всем диапазоне полей, $b \in (0,1)$. Это означает, что конфигурацию скирмиона в 3D следует рассматривать как метастабильную. Хотя этот факт не мешает нам определить спектр магнонов ниже, он может поставить под сомнение такую процедуру. Отметим здесь, что обсуждаемые квантовые поправки к основному состоянию понижают энергию конфигурации скирмионов и, в конечном итоге, могут сделать ее предпочтительной для системы. Подтверждение последнего утверждения требует, однако, расчета квантовой поправки к состоянию спирали, что выходит за рамки настоящего исследования.


\pagebreak
\section{ Уравнение на спектр спиновых волн в базисе функций Ландау }

\subsection{ Гамильтониан на скирмионном кристалле и калибровочный потенциал }

Гамильтониан (\ref{eq:HamQuantBose}) удобно представить в следующем матричном виде, используя введенные ранее безразмерные единицы измерения:
\begin{equation}
\label{eq:HamQuantBoseDimensionless}
\hat {\cal H}_{q} =\begin{pmatrix} F(\hat  L_{z}) &g \\ g & F(  - \hat  L_{z} ) \end{pmatrix}
\end{equation}
где мы ввели следующие определения
$$
\begin{aligned} 
F  (\hat  L_{z} )&\equiv   - \nabla^{2}  + \frac{1 + 3\cos 2\beta }{4r^2} - \frac{2\cos \beta }{r^2}\hat{L}_{z} - \frac{1}{2}\left( \frac{d\beta }{dr} \right)^2  \\
& +  \left\{  - \frac{3\sin 2\beta }{2r} + \frac{2\sin \beta }{r}\hat{L}_{z}  - \frac{d\beta }{dr} \right\} + b\cos \beta   \\ 
g &\equiv \frac{1}{2}\left(  - \frac{\sin ^2\beta }{r^2} + \left( \frac{d\beta }{dr} \right)^2 \right) 
+ \left\{ \frac{d\beta }{dr} - \frac{\sin 2\beta }{2r} \right\}
\end{aligned}  
$$

Данное выражение было выведено для одного скирмиона, расположенного в начале координат, на диске оптимального радиуса $R_0$. Мысленно заполняя плоскость одинаковыми дисками, мы можем предположить, что полный гамильтониан \eqref{eq:HamQuantBose} одиночного скирмиона также может быть продолжен на всю плоскость, если продолжить потенциальные энергии периодическим образом на всю решетку скирмионов. Данное предположение является приближением к реальному гамильтониану скирмионного кристалла. Мы полагаем, что оно является оправданным, так как структура классического решения для гексагональной решетки скирмионов, получаемая другими методами \cite{Mochizuki2012}, практически неотличима от кристалла, получаемого замощением плоскости одиночными скирмионами.

Для дальнейшего анализа оказывается удобным ввести калибровочный потенциал:

\begin{equation}
(- i \nabla - \mathbf{A} ) ^ 2 = - \nabla ^ 2 + 2 i \mathbf{A} \cdot \nabla + i \text{div} \mathbf{A}  + \mathbf{A}^2
\end{equation}

$$
\mathbf{A} = A(r) \mathbf{e}_\phi ,  \qquad  \hat{L}_{z} = - i r \nabla_\phi
$$
где $A(r) = \frac{ \cos\beta } { r } - \sin \beta$ и плотность потока принимает вид:
$$
{\cal B}  = \frac 1 r \nabla_r \left(r A \right) = 
% - \frac 1 r \left( \left( \sin \beta + r \cos \beta \right) \beta' + \sin \beta \right)
r^{-1}  \nabla_r \left(\cos\beta   -  r  \sin \beta \right)  
$$
Выражение \eqref{eq:HamQuantBoseDimensionless} принимает следующий вид
\begin{equation}
\hat {\cal H}_{q} =\begin{pmatrix} (- i \nabla - \mathbf{A} ) ^ 2 + h_1 & g \\ g & ( i \nabla - \mathbf{A} ) ^ 2 + h_1 \end{pmatrix}
\end{equation}
где
\begin{equation}
h_1(r) \equiv  -\frac{\sin^2 \beta }{2r^2} - \frac{1}{2}\left( \frac{d\beta }{dr} \right)^2 +  \left\{  - \frac{\sin 2\beta }{2r}  - \frac{d\beta }{dr} \right\} + b\cos \beta - \sin ^ 2 \beta
\end{equation}
Если мы рассмотрим калибровочное преобразование для введеного поля 
$$
\mathbf{A} \rightarrow \mathbf{A} + \nabla f(\mathbf{r})
$$
$$
\psi \rightarrow \psi e ^ {- i f(\mathbf{r})}
$$
с некоторой функцией  $f = P \varphi \quad P \in \mathbb{N}$, вектор-потенциал примет следующий вид:
$$
\mathbf{A'}= \left(\frac {P + \cos{\beta}} {r} - \sin{\beta} \right) \mathbf{e}_\varphi,
$$
а волновая функция
$$
\psi' = \psi e^{-i P \varphi}.
$$
Гамильтониан, в результате упомянутого калибровочного преобразования, будет иметь следующий вид
\begin{equation}
\label{eq:init_ham_landau}
\hat {\cal H}_{q} \vec{\Psi} = \tau_z E \vec{\Psi}
\end{equation}
$$
\hat {\cal H}_{q} =\begin{pmatrix} (- i \nabla - \mathbf{A} ) ^ 2 + h_1 (r) &  g (r) e^{2 i P \varphi} \\  g (r) e^{-2 i P \varphi}   & ( i \nabla - \mathbf{A} ) ^ 2 + h_1 (r) \end{pmatrix}, \vec{\Psi} = \begin{pmatrix}  \psi_1 \\ \psi_2 \end{pmatrix}
$$
Замечу здесь, что выбор $P=1$ удобен тем, что $A' \sim r^2, r \rightarrow 0$ и вектор-потенциал является непрерывным в нуле. Это видно из асимптотик \eqref{eq:asympt_beta}. В дальнейшем будем считать, что мы выбрали именно эту калибровку с $P=1$.

Таким образом, получается обобщенное уравнение на собственные функции \eqref{eq:init_ham_landau}, с условием нормировки на волновую функцию:
$$
\vec{\Psi}^\dagger \tau_z \vec{\Psi} =|\psi_1|^2-|\psi_2|^2=1
$$

Данное условие вытекает из свойств преобразования Боголюбова для бозонов. Мы опустили ряд деталей в выводе уравнения, считая его известным.\cite{garst}


\subsection{ Выделение затравочной и основной части в гамильтониане }

Мы заметили ранее, что поле $\cal B$ направлено вдоль оси $\mathbf{e}_z$ и зависимость от $r$ дается выражением  ${\cal B} = r^{-1} d(r A(r))/dr$, Рис. \ref{fig:B_top}. Данное поле обладает тем свойством, что поток его через единичную скирмионную ячейку оказывается равен $ -4\pi $ и пропорционален топологическому заряду  $Q$, как видно в \eqref{eq:topChargeSimplify}:


\begin{figure}[t]
\centering	
\includegraphics[width=0.9\columnwidth]{images/b_dependence.pdf}
\caption{Зависимость поля $\cal B$ от $r$ на диске. }
\label{fig:B_top}
\end{figure}


\begin{equation}
\Phi = \int \limits_\Omega {\cal B} \mathbf{e}_z d\mathbf{S} = \int_\phi d \phi \int_r d r \nabla_r \left(\cos\beta   -  r  \sin \beta \right)  = 4 \pi Q
\end{equation}

Теперь, замечая это свойство потока, мы разобъем его на две части, одну, отвечающаю однородному полю с потоком $4 \pi Q$ и вторую с нулевым потоком, для этого вычтем поток однородного поля ${\cal B}_{0} = 4/R_{0}^{2}$  из полной плотности потока  $\cal B$. Оставшаяся часть ${\cal B}_{1} = {\cal B} - {\cal B}_{0}$, очевидно, будет обладать нулевым потоком.
 
Эта операция на самом деле соответствует тому, что калибровочный потенциал в гамильтониане \eqref{eq:init_ham_landau} разбивается на две части $A_0,A_1$:
\begin{equation}
\begin{aligned}
\mathbf{A}  & = \left(A_0 + A_1  \right) \mathbf{e}_\phi, \quad A_0 =   \frac 1 2 r {\cal B}_{0} ,    \\
\quad A_1 & = \frac{1+  \cos\beta } { r } - \sin \beta - \frac 1 2 r {\cal B}_{0}     \\
\end{aligned}
\label{eq:devide_2_hamilte}
\end{equation}
Заметим, по построению $A_{1}(0) = A_{1}(R_{0}) = 0 $ и видно, что $(\pm i \nabla - \mathbf{A} ) ^ 2$  в гамильтониане \eqref{eq:init_ham_landau} принимает вид
\begin{equation}
\begin{aligned}
(\pm i \nabla - \mathbf{A} ) ^ 2 & = (\pm i \nabla - \mathbf{A}_0 ) ^ 2 - 
2 \mathbf{A}_1  (\pm i \nabla - \mathbf{A}_0 )+ \mathbf{A}_1^2   \\
\end{aligned}
\end{equation}
поскольку $ \mbox{div } \mathbf{A}_1=0$.

Можно выделить слагаемое гамильтониана \eqref{eq:init_ham_landau} как невозмущённую часть
$$
\begin{pmatrix} ( -i \nabla - \mathbf{A}_0 ) ^ 2 & 0 \\ 0 & ( i \nabla - \mathbf{A}_0 ) ^ 2 \end{pmatrix}
$$
и возмущенную
$$
\begin{pmatrix} \mathbf{A}_1  (i \nabla - \mathbf{A}_0 ) + u(r) & g (r) e^{2 i  \varphi} \\ g (r) e^{-2 i  \varphi} & \mathbf{A}_1  (-i \nabla - \mathbf{A}_0 ) + u(r) \end{pmatrix}
$$ 
где $u(r) \equiv  h_1(r) + \mathbf{A}_1^2$. 

Решение для невозмущённой части может быть легко получено в базисе собственных функций Ландау\cite{landau}:
\begin{equation}
\label{eq:basis_func_landau}
\ket{n_r, m, \alpha} = \frac{e^{(-1)^{\alpha+1}  i m \phi}} {\sqrt{2 \pi}} \Upsilon_{m,n_r} (r)
\end{equation}
как
$$
\bra{n_r', m', \beta} \hat h (r)  \ket{n_r, m, \alpha} = 2 {\cal B}_{0} \left( n_r + \frac{|m| - m + 1}{2} \right) \delta_{\alpha \beta} \delta_{n_r n'_r} \delta_{m m'} 
$$
\noindent где  
\begin{equation}
\begin{aligned}
\Upsilon_{m, n_r} (r) & = {\cal B}_{0}^{\frac{1+|m|}{2}} \sqrt{\frac{(|m| + n_r)!}{2^{|m|}n_r!(|m|!)^2}} e^{-\frac{{\cal B}_{0} r^2}{4}} r^{|m|} F\left(-n_r,|m| + 1,\frac{{\cal B}_{0} r^2 }{2} \right) 
\end{aligned}
%\label{}
\end{equation}
\noindent $F$  --- это гипергеометрическая функция с условием ортогональности 
$$ \int\limits_0^\infty \Upsilon_{m, n_r}^2 (r) r dr = 1$$
и $n_r \in \mathbb{Z}_+$, $m \in \mathbb{Z}$.

Оставшиеся части гамильтониана \eqref{eq:init_ham_landau}, $\mathbf{A}_1^2$, $g(r)$, $u(r)$ и особое слагаемое 
$\mathbf{A}_1  (\pm i \nabla - \mathbf{A}_0 )$ могут быть рассмотрены как возмущение основного состояния.


\begin{figure}[ht]
\centering	
\includegraphics[width=0.9\columnwidth]{images/potentials_h.pdf}
\caption{Зависимости потенциальных функций \eqref{eq:periodic_potentials} на диске.}
\label{fig:potentials_h}
\end{figure}


\subsection{Матричное уравнение на спектр магнонов на скирмионном кристалле}

Матричное уравнение Шредингера в базисе \eqref{eq:basis_func_landau} принимает следующий вид
\begin{equation}
\label{eq:shred_final}
\mel{m', n'_r, \alpha}{\hat h^{\alpha \beta}(\mathbf{r}) + \hat v^{\alpha \beta}(\mathbf{r}) + \hat u^{\alpha \beta}(\mathbf{r}) + \hat g^{\alpha \beta} (\mathbf{r})) }{n_r,m, \beta} = {\tau_z^{\alpha \beta} \delta_{n_r n'_r} \delta_{m m'} E_{n_r m} }
\end{equation}

\noindent где:
\begin{equation}
\label{eq:definitions}
\begin{aligned}
\hat h^{\alpha \beta}(\mathbf{r})   &= \left( (-1)^\alpha i \nabla - \mathbf{A}_0 \right) ^ 2 \delta_{\alpha \beta} \\
\hat v^{\alpha \beta}(\mathbf{r})   &= - 2 \mathbf{A}_1 \left( (-1)^{\alpha} i \nabla - \mathbf{A}_0 \right) \delta_{\alpha \beta} \\
\hat u^{\alpha \beta}(\mathbf{r})   &= u(\mathbf{r})  \delta_{\alpha \beta}\\
\hat g^{\alpha \beta} (\mathbf{r}) &=   g(\mathbf{r})  e^{ (-1)^{\beta} 2  i   \varphi} \tau^{\alpha \beta}_x 
\end{aligned}
\end{equation}

Как упоминалось выше мы предполагаем, что мы продолжили эти функции периодическим образом на гексагональную решетку с вектором $\mathbf{a}_n =   n_1 \hat {e}_1 + n_2 \hat {e}_2  $  где $n_1,n_2 \in \mathbb{Z}$, $|\hat {e}_1 |=|\hat {e}_2 |= 2 \ell \cos{\frac{\pi}{6}}$ как изображено на Рис. \ref{pic:hexagon_geometry}.

Периодические условия на потенциалы имеют вид:
\begin{equation}
\label{eq:periodic_potentials}
\begin{aligned}
\hat v(\mathbf{r} + \mathbf{a}_n) =& \hat v(\mathbf{r}) \\
\hat u(\mathbf{r} + \mathbf{a}_n) =&  \hat u(\mathbf{r}) \\
\hat g(\mathbf{r} + \mathbf{a}_n) =&  \hat g(\mathbf{r})
\end{aligned}
\end{equation}

\begin{figure}[t]
\centering	
\includegraphics[width=0.8\columnwidth]{images/geometry_lattice.pdf}
\caption{Длина правильного шестиугольника равна $\ell$ и вектора $\hat {e}_1$ и $\hat {e}_2$ направлены к центрам ближайших гексагональных ячеек. }
\label{pic:hexagon_geometry}
\end{figure}

\noindent Из теоремы блоха\cite{landau} следует, что волновая функция в периодическом потенциале должна  удовлетворять условию
\begin{equation}
\label{eq:bloch_wave}
\psi_\alpha (\mathbf{r}) =  e^{i \mathbf{k} \mathbf{r}} u_ \alpha (\mathbf{r} ), \qquad u_ \alpha (\mathbf{r} + \mathbf{a}_n) = u_ \alpha (\mathbf{r})
\end{equation}
где $e^{i \mathbf{Q_p} \mathbf{a}_n }=1$

При этом вектора $\mathbf{Q_p}$ обратной решетки могут быть получены, используя следующие выражения
\begin{equation}
\begin{aligned}
\mathbf{Q}_p = p_1 \hat {v}_1 + p_2 \hat {v}_2  &\qquad p_1,p_2 \in \mathbb{Z} \\
\hat{v}_1 = 2 \pi  \frac{ R_{\frac \pi 2} \hat {e}_2}{\hat {e}_1   \cdot R_{\frac \pi 2} \hat {e}_2}  \\
\hat{v}_2 =  2 \pi  \frac{ R_{\frac \pi 2} \hat {e}_1}{\hat {e}_2 \cdot R_{\frac \pi 2} \hat {e}_1}
\end{aligned}
\end{equation}
$R_{\frac \pi 2}$ - матрица поворота на $\pi/2$ в плоскости.


Заметим, что функции $u(\mathbf{r}), g(\mathbf{r}), A_1(\mathbf{r})$ определены только на диске радиуса $R_0$ с центром в начале координат, зависимости от $r$ изображены на Рис. \ref{fig:potentials_h}.


\pagebreak

\section{ Описание численного алгоритма вычисления спектра спиновых волн и анализ результатов  }


\subsection{Точная форма слагаемых матричного гамильтониана}
Вычисление матричных элементов уравнения \eqref{eq:shred_final} напрямую потребует значительных вычислительных ресурсов, однако можно сделать ряд оптимизаций, которые существенно упростят задачу. 

Нам нужно построить периодическое продолжение потенцальных функций $u(\mathbf{r}), g(\mathbf{r}), A_1(\mathbf{r})$. Для эти целей удобно воспользоваться разложением в ряд Фурье. Как известно любая периодическая функция может быть разложена в ряд вида
$$f (\mathbf{r}) = \frac{1}{\sqrt{|\Omega|}} \sum\limits_{\mathbf{Q}_p} e^{i \mathbf{Q}_p  \mathbf{r}} f(\mathbf{Q}_p) $$
с коэфициентами определямыми на диске как
\begin{equation}
\begin{aligned}
f (\mathbf{Q}_p) &= \frac{1} {|\Omega|} \int\limits_\Omega f (  \mathbf{r} ) e^{- i \mathbf{Q}_p \mathbf{r}} d \mathbf{r} \approx \frac{1}{\pi R_0^2}  \int_{r<R_{0}}  f(\mathbf{r})   e^{-i\mathbf{Q}_p  \mathbf{r}}  d \mathbf{r} , \\
&   =\frac{1}{\pi R_0^2}  \int\limits_{0}^{R_0}  \int\limits_{0}^{2 \pi} e^{-i Q_p r \cos{\varphi}} e^{i m \varphi} f^{\text{loc}}(r) r d\varphi dr =\frac{2 (-i)^{(-m)}}{R_0^2}  \int\limits_{0}^{R_0}  J_{m} (-Q_p r ) f^{\text{loc}}(r) r dr     \\
\end{aligned}
% \label{}
\end{equation} 
где $\quad k = \left|\mathbf{k}\right| $ ,
$\Omega$ - гексагональная ячейка $|\Omega| = \pi R_0^2$, $\mathbf{Q}_p$ вектора обратной решётки и $Q_p=|\mathbf{Q}_p|$. В выводе этой формулы использовалось разложение Якоби — Ангера:
$$
e^{i z \cos \theta} \equiv \sum_{n=-\infty}^{\infty} i^n\, J_n(z)\, e^{i n \theta}
$$

Используя эти формулы для разложение в ряд Фурье соответствующих потенциальных функций \eqref{eq:periodic_potentials}, мы приходим к выводу, что для получения матричных элементов нам нужно будет посчитать следующие величины

\begin{equation}
\label{eq:matrix_elem_k}
\begin{aligned}
&\mel{m', n'_r, \alpha} {e^{i \mathbf{Q}_p\mathbf{r} + i n \varphi} } {n_r,m, \beta} = \\ &\mel{\Upsilon_{m', n'_r}} {\frac{1}{2 \pi}\int _{0}^{ 2 \pi } e^{i Q_p r \cos{\left( \varphi - \varphi_p  \right)} + i \left((-1)^{\beta+1}m + (-1)^{\alpha} m'  + n \right)  \varphi}    d\varphi} {\Upsilon_{m, n_r}} =  \\
& \mel{\Upsilon_{m', n'_r}} {\frac{  e^{i \left(-m_{\alpha \beta}  + n \right)  \varphi_p}}{2 \pi}\int _{0}^{ 2 \pi } e^{i Q_p r \cos{\left( \varphi - \varphi_p  \right)} + i \left(-m_{\alpha \beta}  + n \right)  (\varphi-\varphi_p) }    d\varphi} {\Upsilon_{m, n_r}} =\\
& i^{m_{\alpha \beta} - n} e^{-i (m_{\alpha \beta} - n)  \varphi_p}  \mel{\Upsilon_{m', n'_r}} {J_{m_{\alpha \beta} - n } (Q_p r) }{\Upsilon_{m, n_r}}
\end{aligned}
\end{equation}
где $n \in\mathbb{Z}$, $\varphi_p$ - это угол $\mathbf{Q}_p$ в полярных координатах и 
\begin{equation}
\label{eq:m_alpha_beta}
m_{\alpha \beta} = (-1)^{\beta}m + (-1)^{\alpha+1} m'
\end{equation}
Отметим, что данное выражение уже не зависит от потенциалов нашей задачи.

Выясним далее точный вид для каждого слагаемого в \eqref{eq:shred_final}. 

\subsubsection{Невозмущенная часть: $\hat h$}
Выражение для невозмущенной части мы уже знаем, но выпишем его для удобства ещё раз:
\begin{equation}
\mel{n'_r, m', \alpha} {\left((-1)^\alpha i \nabla - \mathbf{A}_0 \right) ^ 2 \delta_{\alpha \beta}} {n_r, m, \beta} = 
2 {\cal B}_{0} \left( n_r + \frac{|m| - m + 1}{2} \right) \delta_{\alpha \beta} \delta_{n_r n'_r} \delta_{m m'} 
\end{equation}


\subsubsection{Возмущенная часть: специальное слагаемое $\hat v$}
Для  $\hat v$ части \eqref{eq:shred_final} удобно воспользоваться следующими соотношениями:
\begin{equation}
\begin{aligned}
-2 \mathbf{A}_1  (-i \nabla - \mathbf{A}_0 )& = \sqrt{2} \left( A_{1} (\mathbf{r}) e^{ i \varphi  } d^{\dagger} +  A_{1} (\mathbf{r}) e^{-i \varphi  } d \right), \quad \alpha = 1 \\
-2 \mathbf{A}_1  (i \nabla - \mathbf{A}_0 ) & = \sqrt{2} \left( A_{1} (\mathbf{r}) e^{  i \varphi  } c^{} +  A_{1} (\mathbf{r}) e^{-i \varphi  } c^{\dagger} \right) , \quad \alpha = 2
\end{aligned}
\end{equation}
где 
$$
A^{\pm}_{1}   = A_{1}^{x} \pm i A_{1}^{y} = \pm i A_1 (r) e^{ \pm i \varphi  } 
$$
операторы $c,d$ и $c^{\dagger},d^{\dagger} $ являются операторами понижения и повышения уровня Ландау и имею следующий вид в полярных координатах:
\begin{equation}
\begin{aligned}
d &= \frac{e^{i \varphi}} {\sqrt{2}} \left(  \frac {\partial }{\partial r} + i \frac{1}{r}  \frac {\partial }{\partial \varphi} + \frac 1 2   r \cal B  \right) \\
&d^{\dagger} = \frac{e^{ -i \varphi}} {\sqrt{2}} \left(  - \frac {\partial }{\partial r} + i \frac{1}{r}  \frac {\partial }{\partial \varphi} + \frac 1 2    r \cal B \right) \\
&c = \frac{e^{- i \varphi}} {\sqrt{2}} \left(  \frac {\partial }{\partial r} - i \frac{1}{r}  \frac {\partial }{\partial \varphi} + \frac 1 2   r \cal B  \right) \\
&c^{\dagger} = \frac{e^{ i \varphi}} {\sqrt{2}} \left(  - \frac {\partial }{\partial r} - i \frac{1}{r}  \frac {\partial }{\partial \varphi} + \frac 1 2    r \cal B \right) \\
\end{aligned}
\end{equation}
Тогда легко проверить, что:
\begin{equation}
\begin{aligned}
&\mel{n_r',m',\alpha} {\hat v^{\alpha \beta}(\mathbf{r}) }{n_r, m,\beta} =
 \delta_{\alpha \beta} \sqrt{ {2{\cal{B}}_0} } \Big( \\
&\bra{n_r',m',\alpha} (-1)^{\Theta(m-1)}  \sqrt{n_r + 1 + \frac{|m| - m}{2}  }  A_{1} (\mathbf{r}) e^{ -(-1)^{\alpha} i \varphi}   \ket{n_r+\Theta(m-1), m-1,\beta}  \\
&+ \bra{n_r',m',\alpha} (-1)^{\Theta(m)} \sqrt{  n_r + \frac{|m| - m}{2}  }  A_{1} (\mathbf{r}) e^{(-1)^\alpha i \varphi} \ket{n_r-\Theta(m), m +1 ,\beta}   \Big) \\
\end{aligned}
\end{equation}

\noindent где $\Theta$ - функция Хевисайда
$$
\Theta(n)=\begin{cases} 0, & n < 0, \\ 1, & n \ge 0, \end{cases},
$$
и
\begin{equation}
\begin{aligned}
d \ket{0, m'} &= 0 \\
c \ket{0, -m'} &= 0 \\
\end{aligned}
\end{equation}
\noindent и $m \ge 0$.



\noindent окончательная формула принимает следующий вид
\begin{equation}
\label{eq:final_v}
\begin{aligned}
&\mel{m', n'_r, \alpha} { - 2 \mathbf{A}_1 \left( (-1)^{\alpha} i \nabla - \mathbf{A}_0 \right) \delta_{\alpha \beta}} {n_r, m, \beta} \\
&= \delta_{\alpha \beta} i^{m_{\alpha \beta} }  \sqrt{ {2{\cal{B}}_0} } \sum\limits_{p}  A_{1} (Q_p) e^{- i m_{\alpha \beta}\varphi_p} \Big(  \\
&(-1)^{\Theta(m)} \sqrt{  n_r + \frac{|m| - m}{2}  }   \mel{\Upsilon_{m', n'_r}} {J_{m_{\alpha \beta}} (Q_p r)} {\Upsilon_{m +1, n_r-\Theta(m)}}  \\
&  + (-1)^{\Theta(m-1)} \sqrt{n_r + 1 + \frac{|m| - m}{2}  }   \mel{\Upsilon_{m', n'_r}} {J_{m_{\alpha \beta}} (Q_p r)} {\Upsilon_{m-1, n_r+\Theta(m-1)}} 
\Big)
\end{aligned}
\end{equation}

При этом коэфциенты $A_{1} (Q_p)$ определяются следующим образом:
\begin{equation}
\label{eq:fourier_v}
A_{1} (Q_p) = \frac{2}{R_0^2}  \int\limits_{0}^{R_0}  J_{0} (Q_p r ) A_1(r) r dr 
\end{equation} 

Асимптотики $A_1(r)$, как следует из \eqref{eq:asympt_beta} имеют вид
\begin{equation}
\begin{aligned}
A_1(r) &= -\frac{1}{2} r \left|b c_1^2-2 \sqrt{b} c_1-{\cal{B}}_0\right|+O\left(r^3\right), \qquad &r \rightarrow 0 \\
A_1(r) &= -\frac{{\cal{B}}_0 r}{2}+O\left(\frac{1}{r}\right), \qquad &r \rightarrow \infty
\end{aligned}
\end{equation}
\subsubsection{Возмущенная часть: диагональное слагаемое $\hat u$}
Для  $\hat v$ части \eqref{eq:shred_final} мы получаем уравнение в общем виде:
\begin{equation}
\label{eq:final_u}
\mel{m', n'_r,\alpha} {u(\mathbf{r})  \delta_{\alpha \beta}}{n_r,m, \beta} = i^{m_{\alpha \beta} }  \delta_{\alpha \beta} \sum\limits_{p}      e^{- i m_{\alpha \beta} \varphi_p} u(\mathbf{Q}_p)\mel{\Upsilon_{m', n'_r}} {J_{m_{\alpha \beta}} (Q_p r) }{\Upsilon_{m, n_r}}
\end{equation}

\begin{equation}
\label{eq:fourier_u}
u (\mathbf{Q}_p) \approx \frac{2}{R_0^2}  \int\limits_{0}^{R_0}  J_0 (Q_p r ) u(r) r dr 
\end{equation} 
и мы знаем как функции $u$ представлена на диске:
\begin{equation}
u(r) = -\frac{\sin^2 \beta }{2r^2} - \frac{1}{2}\left( \frac{d\beta }{dr} \right)^2 +  \left\{  - \frac{\sin 2\beta }{2r}  - \frac{d\beta }{dr} \right\} + b\cos \beta - \sin ^ 2 \beta + A^2_1(r)
\end{equation}
с асимтоптиками:
\begin{equation}
\begin{aligned}
u(r) &= \widetilde{c}_0 - \widetilde{c}_1 r^2 + O(r^3), \qquad &r \rightarrow 0 \\
u(r) &= \frac{1}{4} r^2 {\cal{B}}_0^2 + o(1), \qquad &r \rightarrow \infty
\end{aligned}
\end{equation}
с некоторыми константами $\widetilde{c}_0 > 0$ и $\widetilde{c}_1 > 0$.

\subsubsection{Возмущенная часть: внедиагональное слагаемое $\hat g$}

\begin{equation}
\label{eq:final_g}
\begin{aligned}
&\mel{m', n'_r, \alpha} {g(\mathbf{r})   e^{ (-1)^{\beta} 2  i   \varphi} \tau^{\alpha \beta}_x} {n_r,m,\beta} = \\ 
&i^{m_{\alpha \beta}} \tau^{\alpha \beta}_x \sum\limits_{p}      e^{-i m_{\alpha \beta} \varphi_p}  g(\mathbf{Q}_p)\mel{\Upsilon_{m', n'_r}} {J_{m_{\alpha \beta}} ( Q_p r) }{\Upsilon_{m, n_r}}
\end{aligned}
\end{equation}
\noindent где 
\begin{equation}
\label{eq:fourier_g}
g (\mathbf{Q}_p) \approx \frac{1}{\pi R_0^2}  \int  g(\mathbf{r})   e^{ (-1)^{\beta} 2  i   \varphi} e^{-i \mathbf{Q}_p \mathbf{r}}  d\mathbf{r} = - \frac{2}{R_0^2}  \int\limits_{0}^{R_0}  J_2 (Q_p r ) g(r) r dr 
\end{equation} 
и функция $g$ на диске имеет вид:
\begin{equation}
g(r) = \frac{1}{2}\left(  - \frac{\sin ^2\beta }{r^2} + \left( \frac{d\beta }{dr} \right)^2 \right) 
+ \left\{ \frac{d\beta }{dr} - \frac{\sin 2\beta }{2r} \right\}
\end{equation}
с асимтоптиками:
\begin{equation}
\begin{aligned}
g(r) &= -\frac{\left|b^2 c_1^4-4 b^{3/2}
   c_1^3\right|}{6} r^2  +O\left(r^3\right), \qquad &r \rightarrow 0 \\
g(r) &= -\frac{c_2 b^\frac{1}{4}}{\sqrt{r}} e^{-\sqrt{b} r} + O\left(\frac{e^{-\sqrt{b} r}}{r^\frac{3}{2}}\right), \qquad &r \rightarrow \infty
\end{aligned}
\end{equation}



\subsection{ Преобразование обобщенного уравнения на собственные числа в стандартную форму  }


Обобщенное уравнение \eqref{eq:init_ham_landau} может быть переписано в стандартной форме, если ввести матрицу
$$
\tau_z^{1/2} = 
	\begin{pmatrix} 	
		1 &  0 \\
		0 &  i 
	\end{pmatrix}
$$
которая обладает свойствами $(\tau_z^{1/2})^2 = \tau_z$ и $\tau_z^{-1/2}=(\tau_z^{1/2})^{-1}=(\tau_z^{1/2})^\dagger$.
Используя эту матрицу и снова переопределяя вектор 
$$
\vec{\Psi}'=\tau_z^{1/2} \vec{\Psi}
$$
и модифицируя уравнение
$$
\tau_z^{-1/2} \hat {\cal H}_{r} \tau_z^{-1/2} \vec{\Psi}' = E \vec{\Psi}'
$$
мы приходим к стандартному уравнению на собственные числа:
\begin{equation}
\label{eq:std_form_eigen}
\hat {\cal H'}_{r} = \tau_z^{-1/2} \hat {\cal H}_{r} \tau_z^{-1/2} = \begin{pmatrix} 
(- i \nabla - \mathbf{A} ) ^ 2 + h_1 (r) &  -i g (r) e^{2 i C \varphi} \\  
-i g (r) e^{-2 i C \varphi}   & - ( i \nabla - \mathbf{A} ) ^ 2 - h_1 (r) 
\end{pmatrix}
\end{equation}

$$
\hat {\cal H'}_{r} \vec{\Psi}' = E \vec{\Psi}'
$$
Решая данное уравнение, мы можем получить решение исходного уравнения \eqref{eq:init_ham_landau}, произведя замену
$$
\vec{\Psi} = \tau_z^{-1/2} \vec{\Psi}'
$$
Гораздо удобнее работать с уравнением именно в стандартной форме и диаганализация всегда подразумевается для матрицы построенной именно с использованием формулы \eqref{eq:std_form_eigen}.


\subsection{ Суммирование на гексагональной решетке }

Можем заметить, что на гексагональной решетке некоторые центры элементарных ячеек расположены на одинаковом расстоянии от начала координат. Если провести к центрам таких ячеек вектора, то окажется, что число $w$ векторов одинаковой длины всегда будет кратно шести. Предположим, что некоторая функция $f$ обладает симметрией относительно вращений $f(\mathbf{k})=f(k)$, тогда сумма по всем векторам $\mathbf{Q}_p$ гексагональной решетки может быть существенно упрощена:
\begin{equation}
\label{eq:sum_qp}
\begin{aligned}
&\sum\limits_{p} e^{i m_0\varphi_p}   f(Q_p)  = \sum\limits_{p_0} e^{im_0\varphi_{{p_0}}} f(Q_{p_0})   \sum\limits_{n=0}^{w-1} e^{i m_0 n \frac{2 \pi}{w}} \\
  &=\sum\limits_{p_0} w e^{i m_0\varphi_{p_0}} f(Q_{p_0}) \left[m_0 = 0 \mod w  \right] 
\end{aligned}
\end{equation}
где подразумевается, что $w$ - это количество равноудаленных от начала координат центров гексагональной решетки с длиной $Q_{p_0}$, в этом выражении используется скобка Айверсона, которая раскрывается следующим образом
$$
\left[m_0 = 0 \mod w  \right] =   
\begin{cases}
   1 & m_0 = 0 \mod w \\
   0 & m_0 \neq 0 \mod w 
 \end{cases}
$$
$\mathbf{Q}_{p_0}$ - это один из $w$ векторов длины $Q_{p_0}$ с наименьшим углом $\varphi_{p_0}$.
Эта формула используется для вычисления сумм в матричных элементах \eqref{eq:final_g}, \eqref{eq:final_u} и \eqref{eq:final_v}.

\begin{center}
\begin{tabular}{ |c|c|c| }
 \hline
 $w$  & $Q_{p_0}$, $4 \pi / 3 \ell$ & $\varphi_{p_0}$\\ 
 \hline
 1&0&0\\
 6&1&0\\
 6&1.732&1.570796\\
 6&2&0\\
 12&2.645751&1.380671\\
 6&3&0\\
 6&3.464102&1.570796\\
 12&3.605551&1.289761\\
 6&4&0\\
 12&4.358899&1.455835\\
 12&4.582576&1.237323\\
 \hline
\end{tabular}
\captionof{table}{Пример первых значений $Q_{p_0}$, $\varphi_{p_0} $ и $w$.} 
\end{center}


\subsection{ Численный алгоритм вычисления }

Теперь мы можем явно выписать алгоритм (Алг. \ref{alg:the_alg}) для вычисления спектра и волновых функций в базисе \eqref{eq:basis_func_landau}. Следует отметить, что самой сложной частью алгоритма, с вычислительной точки зрения, является часть относящаяся к получению  значений матричных элементов $\mel{\Upsilon_{m', n'_r}} {J_{m_{\alpha \beta}} (Q_p r) }{\Upsilon_{m, n_r}}$, которые, тем не менее, являются некоторыми числами, несвязанными с постановкой задачи в данной работе. 

\begin{algorithm}[ht]
 \SetAlgoLined
 Фиксируем $m_\text{max}$, $n_\text{max}$ и $p_\text{max}$

Решаем уравнение \eqref{eq:eulerLagr} и строим интерполяцию $\beta$

\For{$p_0\gets0 \quad \KwTo \quad p_\text{max} \quad$}{
			Вычисляем набор чисел $w$, $Q_{p_0}$, $\varphi_{p_0}$
			
    		    Вычисляем коэффициенты  ряда Фурье $A_{1} (Q_{p_0})$, $ u(Q_{p_0})$  и $g(Q_{p_0})$ по соответствующим формулам \eqref{eq:fourier_v}, \eqref{eq:fourier_u} и \eqref{eq:fourier_g}
        }
\For{$p_0\gets0 \quad \KwTo \quad p_\text{max} \quad$}{
	\For{$n,n' \gets0 \quad \KwTo \quad n_\text{max} \quad$}{
		\For{$m,m' \gets -m_\text{max} \quad \KwTo \quad m_\text{max} \quad$}{
			Вычисляем 
			
			$\mel{\Upsilon_{m', n'_r}} {J_{m_{\alpha \beta}} (Q_{p_0} r) }{\Upsilon_{m, n_r}}$ если $m_{\alpha \beta} = 0 \mod 6$
		}
	}
}
\For{$n,n' \gets0 \quad \KwTo \quad n_\text{max} \quad$}{
		\For{$m,m' \gets -m_\text{max} \quad \KwTo \quad m_\text{max} \quad$}{
			Вычисляем матричные элементы по формулам \eqref{eq:final_v}, \eqref{eq:final_u} и \eqref{eq:final_g}, при этом используя формулу \eqref{eq:sum_qp} для суммирования по $p_0$ от $0$ до $p_\text{max}$
		}
}
Используя вычисленные матричные элементы и уравнение \eqref{eq:std_form_eigen}, строим матрицу $\mel{m', n'_r, \alpha}{\hat {\cal H'}_{r}}{n_r,m, \beta}$ размера $N_\text{max} \cross N_\text{max}$, где  $N_\text{max}=2 (2 m_\text{max} + 1) (n_\text{max} + 1)$

Диагонализируем полученную матрицу 

\caption{Алгоритм вычисления спектра спиновых волн на скирмионном кристалле}\label{alg:the_alg}
\end{algorithm}

В приведенном алгоритме (Алг. \ref{alg:the_alg}) опущен ряд деталей, связанных с оптимизацией, к примеру, нет необходимости вычислять все матричные элементы т. к. матрицы являются эрмитовыми и достаточно вычислить лишь элементы на главной диагонали и выше её.

\pagebreak
\subsection{ Результаты рассчётов по предложенному алгоритму }

В рамках данной работы был реализован предложенный алгоритм (Алг. \ref{alg:the_alg}) и были произведены необходимые расчёты, при этом использовались следующие значения $m_\text{max}=18$ и $n_\text{max}=10$ для расчёта. Результирующий спектр приведен на Рис.  \ref{fig:landau_spectrum}. Мы получаем дискретный спектр в силу того, что мы ограничиваем при численном анализе полный базис конечным, что в каком-то смысле, соответствует помещению системы в потенциальный "ящик".

\begin{figure}[t]
\centering	
\includegraphics[width=0.95\columnwidth]{images/spectrum_landau.pdf}
\caption{Результирующий дискретный спектр, полученный после диагонализации матрицы \eqref{eq:shred_final}. На графике видно, что уровни энергии сгущаются вокруг уровней Ландау $n_r + \frac{|m| - m + 1}{2}$. }
\label{fig:landau_spectrum}
\end{figure}

Волновые функции магнонов могут быть получены, используя результирующие собственные вектора матрицы \eqref{eq:definitions} в базисе \eqref{eq:basis_func_landau}.

\begin{equation}
\label{eq:wave_func_final}
\Psi_N = \sum_{n,m,\alpha} C_{n_r, m, \alpha} \ket{n_r, m, \alpha}
\end{equation}
где $C_{n_r, m, \alpha}$ соответствующие компоненты вектора.
Характерные примеры вычисленных функций приведены на Рис. \ref{fig:landau_eigenfunction}. Из этих графиков мы можем увидеть, что размер нашего базиса оказываются недостаточно большим для того, чтобы полученная функция была явно периодической. Для получения картины дисперсии магнонов\cite{garst_2017, roldan} нам необходимо теперь соотнести полученные уровни энергии на Рис. \ref{fig:landau_spectrum}  со значениями волнового вектора $\mathbf{k}$ в зоне Бриллюэна. Для этой цели нам необходимо построить зависимость $E_N(\mathbf{k})$ в зоне Бриллюэна, однако потребует достаточно точного вычисления $\Psi_N$.

Данная зависимость может быть получена, если воспользоваться свойствами функции Блоха \eqref{eq:bloch_wave} для функции \eqref{eq:wave_func_final}. Если взять мнимую часть отношения $\Psi_N$ в двух точках, отличающихся на вектора решетки $\mathbf{a}_n$ и $\mathbf{a}_n'$

\begin{equation}
\begin{aligned}
\begin{cases}
	\ln{ \Psi_N(\mathbf{r} + \mathbf{a}_n) } -  \ln{\Psi_N(\mathbf{r})} = i \mathbf{k} \mathbf{a}_n \\
	\ln{ \Psi_N(\mathbf{r} + \mathbf{a}_n') } -  \ln{\Psi_N(\mathbf{r})}  = i \mathbf{k} \mathbf{a}_n'
\end{cases}
\end{aligned}
\end{equation}
Решая данную систему уравнений на компоненты волнового вектора $k_x,k_y$ для заданных $\mathbf{a}_n$, $\mathbf{a}_n'$, таких что $\mathbf{a}_n \nparallel \mathbf{a}_n'$, можно получить $\mathbf{k}$ для данного уровня энергии $E_N$, соответствующего функции $\Psi_N$. При этом выбор точки $\mathbf{r}$ лучше осуществить, используя небольшую $|\mathbf{r}|<R_0$ такую, что $\Psi_N(\mathbf{r}) \neq 0$.


\begin{figure}[t]
\centering	
\includegraphics[width=1.03\columnwidth]{images/eigenfunc.pdf}
\caption{Пример волновой функции, полученной при размере базиса $m_\text{max}=18$ и $n_\text{max}=10$, соответствующий наименьшей энергии. Слева, под буквой A изображена зависимость модуля волновой функции, а под B - зависимость мнимой части. Из рисунка видно, что функция симметрична относительно поворотов на $\pi/6$, что и следовало ожидать исходя из симметрии гамильтониана. Однако можно заметить, что свойство периодичности не наблюдается, в силу ограниченности базиса. }
\label{fig:landau_eigenfunction}
\end{figure}


В рамках данной работы нам не удалось получить корректную дисперсию магнонов с помощью вычисленных волновых функций \eqref{eq:wave_func_final} в силу того, что они являются сильно локализованными вблизи начала координат. Эту проблему мы связываем с тем, что для получения правильных периодических функций Блоха необходимо значительно увеличивать базис. Таким образом, задача получения дисперсии, в рамках данного метода, остается открытой и требует дополнительной работы.


\pagebreak
\specialsection{Заключение}

В работе был продемонстрирован метод для получения спектра спиновых волн на скирмионном кристалле в базисе функций Ландау. Для этих целей сначала использовался метод квазиклассического квантования, с помощью которого было получено уравнение Шрёдингера для одиночного скирмиона, а затем, после введения ряда упрощений, уравнение было продолжено периодическим образом на скирмионный кристалл. Были выведены необходимые формулы для построения решения в базисе функций Ландау и предложен алгоритм вычисления спектра.

Для проведения расчётов была написана соответствующая программа, получен спектр и волновые функции с её помощью. Однако, как уже было отмечено, вычисленные волновые функции оказываются сильно локализованными, и это означает, что для получения дисперсии магнонов на решетке необходимо либо увеличивать базис значительно, что потребует больших вычислительных ресурсов, либо искать способы стабилизации решения. Поскольку часто погрешность определения собственных векторов оказывается больше чем для собственных значений, мы можем ожидать, что спектр получается достаточно точным, даже при небольшом базисе, но у нас нет возможности соотнести его с конкретными значениями волнового вектора.

В дальнейшем предполагается использование полученного алгоритма для получения более точной картины дисперсии магнонов. Это является основой для вычисления наблюдаемых восприимчивостей и тензора магнитной восприимчивости такой системы. Знание этого тензора необходимо для описания экспериментов по рассеянию нейтронов.

Выражаю благодарность за многочисленные полезные обсуждения и замечания своему научному руководителю Д. Н. Аристову.


% Библиография в cpsconf стиле
% Аргумент {1} ниже включает переопределенный стиль с выравниванием слева
\pagebreak
\begin{thebibliography}{1}
\bibitem{ahiezer} Ахиезер А.И. \flqq Спиновые волны\frqq. Наука, 1967

\bibitem{aristov1} Aristov, D.N. et al. \flqq Magnon spectrum in ferromagnets with a skyrmion\frqq. JETP Lett. 102, 2015, pp. 511

\bibitem{belavin} Белавин, А. А. и др. \flqq Метастабильные состояния двумерного изотропного ферромагнетика\frqq. Письма в ЖЭТФ №10 v. 22, 1975, pp. 503-506

\bibitem{bogdanov} Bogdanov, A., et al. \flqq Thermodynamically stable magnetic vortex states in magnetic crystals\frqq. J. Magn. Magn. Mater. 138, 1994, pp. 255-269
 
\bibitem{back} Back, C. H., et al.  \flqq The 2020 Skyrmionics Roadmap\frqq. Journal of Physics D: Applied Physics, 2020

\bibitem{lacrox} Crépieux, A. et al. \flqq Dzyaloshinsky–Moriya interactions induced by symmetry breaking at a surface\frqq.  J. Magn. Magn. Mater. 182, 1998, pp. 341–349

\bibitem{Luo2019} Luo, H. et al. \flqq Strong hopping induced Dzyaloshinskii–Moriya interaction and skyrmions in elemental cobalt\frqq. npj Comput Mater 5 (50), 2019

\bibitem{landau} Ландау, Л. Д. и др. \flqq Квантовая механика (нерелятивистская теория)\frqq. Наука, 1989

\bibitem{fert} Fert, A. et al. \flqq Skyrmions on the track\frqq. Nature nanotechnology v. 8, 2013, pp. 152-155

\bibitem{garst_2017} Garst, M. et al. \flqq Collective spin excitations of helices and magnetic skyrmions: review and perspectives of magnonics in non-centrosymmetric magnets\frqq. Journal of Physics D: Applied Physics IOP Publishing 50, 2017, 293002

\bibitem{nagaosaHan} Han, J. H. et al. \flqq Skyrmion lattice in a two-dimensional chiral magnet\frqq. Phys. Rev. B 82, 2010, 094429

\bibitem{kitaev} Kitaev, A. \flqq Periodic table for topological insulators and superconductors\frqq. AIP Conference Proceedings 1134 (1), 2009, pp. 22-30

\bibitem{kiselevBogdanov} Kiselev, N. S. et al. \flqq Chiral skyrmions in thin magnetic films: new objects for magnetic storage technologies?\frqq. J. Phys. D v. 44, 2011, pp. 4

\bibitem{milde} Milde, P. et al. \flqq Unwinding of a Skyrmion Lattice by Magnetic Monopoles\frqq. Science Vol. 340, 2013, pp. 1076-1080

\bibitem{munzer} Munzer, W. et al. \flqq Emergent electrodynamics of skyrmions in a chiral magnet\frqq. Phys Rev B 81, 2010, pp. 301–304

\bibitem{rosch_muller} Müller J. et al, \flqq Capturing of a magnetic skyrmion with a hole \frqq. Phys. Rev. B 91, 2015, pp. 054410

\bibitem{mulhbauer} Mühlbauer, S. et al. \flqq Skyrmion Lattice in a Chiral Magnet\frqq. Science 323, 2009, pp. 915-919

\bibitem{Maleyev2006} Maleyev, S. V. \flqq Cubic magnets with Dzyaloshinskii-Moriya interaction at low temperature\frqq. Phys. Rev. B  73, 2006, 174402

\bibitem{Mochizuki2012} Mochizuki, M. \flqq Spin-Wave Modes and Their Intense Excitation Effects in Skyrmion Crystals\frqq.  Phys. Rev. Lett. 108, 2012, 017601

\bibitem{nagaosa} Nagaosa, N. et al. \flqq Topological properties and dynamics of magnetic skyrmions\frqq. Nature nanotechnology 8, 2013, pp. 899-909

\bibitem{rosch_pfleiderer} Pfleiderer, C. et al. \flqq Single skyrmions spotted\frqq. Nature 465, 2010, pp. 880–881

\bibitem{rajaraman} Rajaraman, R. \flqq Solitons and instantons\frqq. Amsterdam, Netherlands: North-holland, 1982

\bibitem{roslerBogdanov} Roessler, U. K. et al. \flqq Spontaneous skyrmion ground
states in magnetic metals\frqq. Nature 442, 2006, pp. 797–801

\bibitem{Rybakov2015} Rybakov, F. N. et al. \flqq New spiral state and skyrmion lattice in 3D model of chiral magnets\frqq. New J. Phys. 18, 2016, 045002

\bibitem{roldan} Roldán-Molina, A. S. Nunez et al. \flqq Topological spin waves in the atomic-scale magnetic skyrmion crystal\frqq. New J. Phys. 18, 2016, 045015

\bibitem{skyrme} Skyrme, T. \flqq A unified field theory of mesons and baryons\frqq. Nuclear Physics. 31, 1962, pp. 556–569.

\bibitem{garst} Sch{\"u}tte, C. et. al. \flqq Magnon-skyrmion scattering in chiral magnets\frqq. Phys. Rev. B 90, 2014, 094423 

\bibitem{yu} Yu, X. Z. et al. \flqq Real-space observation of a two-dimensional skyrmion crystal\frqq. Nature 465, 2010, pp. 901-904

\bibitem{zhang} Zhang, X. et al. \flqq Magnetic skyrmion transistor: skyrmion motion in a voltage-gated nanotrack\frqq. Sci. Rep. 5, 2015, 11369

\end{thebibliography}
\end{document}