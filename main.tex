% ОБЯЗАТЕЛЬНО ИМЕННО ТАКОЙ documentclass!
% (Основной кегль = 14pt, поэтому необходим extsizes)
% Формат, разумеется, А4
% article потому что стандарт не подразумевает разделов
% Глава = section, Параграф = subsection
% (понятия "глава" и "параграф" из стандарта)
\documentclass[a4paper,article,14pt]{extarticle}

% Подключаем главный пакет со всем необходимым
\usepackage{spbudiploma}

% Пакеты по желанию (самые распространенные)
% Хитрые мат. символы

\usepackage{euscript}
% Таблицы
\usepackage{longtable}
\usepackage{makecell}
% Картинки
\usepackage[pdftex]{graphicx}

\usepackage{amsthm, amssymb, amsmath, amsfonts, physics}
\usepackage{mathtext,cite,enumerate,float}
\usepackage{textcomp}

\DeclareMathOperator{\Rot}{rot}
\DeclareMathOperator{\Div}{div}
\DeclareMathOperator{\Sp}{Sp}
\begin{document}

% Титульник в файле titlepage.tex
\begin{titlepage}
\newpage

\begin{center}
САНКТ-ПЕТЕРБУРГСКИЙ ГОСУДАРСТВЕННЫЙ УНИВЕРСИТЕТ\\
\vspace{1cm}

%\hrulefill
\end{center}

% \begin{flushright}
% На правах рукописи
% \end{flushright}
% \begin{flushright}
% \includegraphics[width=0.27\linewidth]{pic/signature.pdf}
% \end{flushright}

\vspace{0.5cm}
\begin{center}
ЦЫПИЛЬНИКОВ Андрей Васильевич
\end{center}

\vspace{1cm}
\begin{center}
    \textbf{выпускная квалификационная работа}
\end{center}{}
\vspace{1cm}

\begin{center}
\Large{\bf Спиновые волны в скирмионном кристалле}
\end{center}
\vspace{1cm}
\begin{center}
Направление 03.06.01 «Физика и астрономия» \\
Основная образовательная программа MK.3008.2016 «Физика»
\end{center}
\vspace{1cm}


\begin{flushleft}
\hspace{\stretch{1}} Научный руководитель:\\
\hspace{\stretch{1}} Аристов Дмитрий Николаевич\\
\hspace{\stretch{1}} д. ф.-м. н., проф.\\
\hspace{\stretch{1}} Рецензент:\\
\hspace{\stretch{1}} Демидов Юрий Андреевич\\
\hspace{\stretch{1}} к. ф.-м. н.\\
\vspace{1.5em}
\end{flushleft}

\vspace{\fill}

\begin{center}
Санкт-Петербург -- 2020
\end{center}
\end{titlepage}


% Содержание
\tableofcontents
\pagebreak

\specialsection{Введение}

Изучение различных существенно нелинейных явлений и эффектов в физике вызывает большой интерес как со стороны теоретиков, так и со стороны экспериментаторов.  Особую роль в анализе нелинейных явлений играет топология --- раздел математики, изучающий свойства  математических объектов, сохраняющиеся при непрерывных преобразованиях. Топология находит своё применение в разных областях физики, например для описания солитонов в моделях квантовой теории поля \cite{rajaraman} или классификации топологических изоляторов. \cite{kitaev} В данной работе интерес представляет модель Скирма \cite{skyrme}, которая используется не только в физике высоких энергий, но и в теории магнитоупорядоченных сред. Солитоны данной модели традиционно называются скирмионами. Существование стабильных скирмионов было предсказано в тонких слоях магнитных материалов с нарушением инверсии, например в соединениях структурного типа $B20$.\cite{roslerBogdanov}


Свойства магнитоупорядоченных кристаллов в областях низких температур определяются спиновыми волнами, \cite{ahiezer} могущими распространяться в них. Согласно принципу квантово-волнового дуализм для этих волн существуют связанные ними частицы - магноны, и сами по себе они во многом аналогичны фононам --- квантам звука в обычных твердых телах, однако обладают ещё и ненулевым магнитным моментом. Теория магнонов в простейшем типе магнитоупорядоченных кристаллов - ферромагнетиках – изучена достаточно хорошо. Подобная теория в магнетиках со скирмионным основным состоянием представляет большой интерес, особенно в связи с недавними экспериментальными работами \cite{mulhbauer, yu}, поскольку позволила бы лучше понять свойства скирмионных кристаллов. По большей части данная работа посвящена этой теме.


\begin{figure}[h]
	\centering
	\includegraphics[width=0.5\paperwidth]{images/skyrmionPic.png}
	\caption{Скирмион на гексагональной ячейке}
	\label{pic:skyrmion}
\end{figure}

Скирмионы наблюдаются в экспериментах, как в виде одиночных возбуждений, стабильных за счёт неоднородности в кристалле\cite{rosch_muller}, так и в виде упорядоченной гексагональной решетки в так называемой $A$-фазе \cite{muhlbauer}, при этом 3D скирмионы могут рассматриваться как «стопки» из двумерных гексагональных решёток.\cite{milde}
На сегодняшний день удалось найти немало материалов, в которых обнаружили скирмионную решетку, например, металлы $\mathrm{Mn\,Si}$ и $\mathrm{Fe\,Ge}$, полупроводник $\mathrm{Fe_{1-x}\,Co_x\,Si}$ \cite{munzer} и изолятор $\mathrm{Cu_2\,O\, Se\, O_3}$. Эти вещества обладают различными электронными характеристиками, но одинаковыми магнитными свойствами. \cite{nagaosa}



\specialsection{Постановка задачи}
Целью данной работы является вывод уравнения Шрёдингера на спектр спиновых волн двумерного скирмионного кристалла и численное решение полученного уравнения. В работе подробно описывается предложенный полуаналитический метод для получения спектра и приводятся основные уравнения, необходимые для проведения численного анализа. В рамках метода предлагается способ получения дисперсии магнонов для данной системы.

\specialsection{Обзор литературы}

Одной из первых работ, где были рассмотрены магнитные скирмионы можно считать работу Белавина и Полякова \cite{belavin} ими же была проанализирована стабильность конфигурации скирмиона на диске в классическом случае. Было продемонстрировано, что состояние является метастабильным. В более поздних работах было замечено, что нарушение инверсии в магнитной системе приводит к стабилизации решения.\cite{bogdanov} Здесь стоит отметить, что большинство теоретических работ на эту тему ограничивались лишь классическим рассмотрением динамики скирмионов.\cite{kiselevBogdanov} Однако позже, в работах  \cite{aristov1, garst} был применен метод квазиклассического квантования и проведен анализ спектра магнонов в такой конфигурации. При этом исследовалась конфигурация с одним скирмионом и не рассматривался случай решетки. Анализ дисперсии магнонов на скирмионном кристалле был выполнен в работах \cite{garst_2017, roldan}, где также было показано, что различные состояния таких магнонов на решетке обладают нетривиальными числами Черна. Важно отметить в том числе недавнюю работу \cite{back}, в которой отражены многие актуальные проблемы и ожидания исследователей в физике скирмионов.

Скирмионы интересны не только в фундаментальном аспекте, но и с точки зрения различных технических приложений, например, с ними связывают надежды на создание сверхплотных, долговременных носителей информации \cite{fert} или транзисторов \cite{zhang}. От них ожидают сильного воздействия на спиновый ток. Электрон, движущийся сквозь скирмион, несколько раз меняет спиновую ориентацию, подстраивая ее под локальное распределение намагниченности, в результате чего на него действует эффективная сила, изменяющая направление его движения, что макроскопически должно проявляться как разновидность эффекта Холла.\cite{rosch_pfleiderer}


\pagebreak
\section{Вывод основного гамильтониана для одного скирмиона и построение классического решения}
\subsection{Базовый гамильтониан для двумерного хирального магнетика}

Традиционно магнетики могут быть рассмотрены в рамках макроскопической теории обменной модели Гейзенбeрга, которая вытекает из соответствующей микроскопической теории. Микроскопическая теория при этом вытекает из законов релятивисткой квантовой теории.  Строгий вывод осуществляется путем рассмотрения квантово-механического принципа неразличимости тождественных частиц. Энергия кристалла при этом рассматривается как сумма кулоновских энергий отдельных атомов, образующих периодическую решетку. На волновую функцию кристалла накладываются требования принципа Ферми. Это ведет к тому, что полная энергия начинает зависит от моментов импульса отдельных атомов. Вследствие этого конечная энергия с определенной долей точности может быть получена добавлением к исходному гамильтониану эффективной энергии, зависящей от моментов импульса $\widehat{S}_i^\alpha$ отдельных атомов на решетке. При этом латинский индекс здесь и далее по тексту отвечает пространственным переменным, а греческий - спиновым. Будем считать, что $\widehat {\mathbf{S}}_{i}$ обозначает вектор в спиновом пространстве. При этом подразумевается, что операторы $\widehat{S}$ удовлетворяют коммутационным соотношениям $[\widehat{S}_i^\alpha, \widehat{S}_j^\beta]=i \delta_{ij} \epsilon_{\alpha \beta \gamma} \widehat{S}_i^\gamma $. Используя эти операторы и  обменные интегралы $J(\mathbf{r}_i - \mathbf{r}_j)$, мы можем записать энергию для двух атомов как $J\left(\mathbf{r}_i - \mathbf{r}_j \right){{\widehat {\mathbf{S}}}_i}{{\widehat {\mathbf{S}}}_j}$, при этом будем подразумевать, что функция $J\left(\mathbf{r}_i - \mathbf{r}_j \right)<0$ и $J(\mathbf{r})=J(r)\simeq -e^{-r}$.

Кроме обменного слагаемого будем учитывать и другие вклады в энергию. Поскольку мы рассматриваем двумерную магнитную систему без центра инверсии, то мы включаем в рассмотрение взаимодействие Дзялошинского-Мория, которое может быть представлено для пары спинов как 
$${H_{DM}} = {{\mathbf{D}}_{ij}} \cdot \left( {\widehat {\mathbf{S}}}_{i} \times {{\widehat {\mathbf{S}}}_j} \right), $$
с вектором $\mathbf{D}_{ij} = D \left(\mathbf{r}_i - \mathbf{r}_j \right)$, лежащим в плоскости. \cite{lacrox} Данное взаимодействие есть одно из проявлений спин-орбитального взаимодейстия и является релятивистским эффектом. Учтем ещё Зеемановский вклад, считая, что в системе есть внешнее магнитное поле $\mathbf{H}_e$ направленное перпендикулярно к плоскости. В некоторых работах дополнительно учитывают еще анизотропию типа «легкая ось», \cite{bogdanov} но мы не будем включать её в рассмотрение, поскольку общая картина, после добавления этого слагаемого, существенно не изменяется. Как будет показано далее, подбирая параметры модели с учетом лишь упомянутых выше слагаемых, мы можем получить стабильную конфигурацию скирмионного кристалла. Полный гамильтониан микроскопической теории принимает следующий вид:

\begin{equation}
\label{eq:ham_init}	
H = \sum\limits_{i,j} {\left[ {J\left( {{{\mathbf{r}}_i} - {{\mathbf{r}}_j}} \right){{\widehat {\mathbf{S}}}_i}{{\widehat {\mathbf{S}}}_j} + {{\mathbf{D}}_{ij}}  {{\widehat {\mathbf{S}}}_i} \times {{\widehat {\mathbf{S}}}_j}} \right] - \mathbf{H}_e} \sum\limits_{i} {{{\widehat {\mathbf{S}}}_i}}
\end{equation}

Осуществим переход к макроскопической теории и, для этой цели, определим оператор плотности магнитного момента импульса
\begin{equation}
\label{eq:densSpinOperator}	
{\widehat {\mathbf{S}}} (\mathbf{r}) = \sum_i {{\widehat {\mathbf{S}}}_i} \delta (\mathbf{r} - \mathbf{r}_i)
\end{equation}
и соответствующую макроскопическую плотность магнитного момента импульса
\begin{equation}
\label{eq:densSpinVec}	
{\mathbf{S}}_\mathbf{r} \equiv {\mathbf{S}} (\mathbf{r}) = \frac{1}{v_0} \int\limits_{v_0} {\Sp{\left( {\widehat {\mathbf{S}}} (\mathbf{r'}) {\widehat \rho  } (\mathbf{r'}) \right)} d \mathbf{r'} }
\end{equation}
где под ${ \widehat \rho }$ мы понимаем матрицу плотности рассматриваемого вещества. Интегрирование в \eqref{eq:densSpinVec} производится в пределах физического малого объема $v_0$ с центром в точке $\mathbf{r}$. Будем считать при этом, что $a_0^3  \ll v_0 \ll \ell^3 $, где $a_0$ - межатомное расстояние, а $\ell$ - некоторый характерный размер неоднородности, в данном случае - это постоянная сверхрешетки скирмионов. Значение оператора спина $s$ предполагается большим для большинства реальных магнетиков и это означает, что $\widehat{\mathbf{S}}^{2}({\mathbf{r}}) = s(s+1) \gg 1$. В пределе $s\to \infty$, магнитный вектор становится неотличим от классического вектора намагниченности с длиной $\mu = s/v_{0}$.


Теперь, усредняя вектора \eqref{eq:ham_init}, мы заменяем произведение операторов произведением средних локальных плотностей (\ref{eq:densSpinVec}), являющихся $c$-числами, и получаем следующее выражение для энергии
\begin{equation}
\label{eq:ham}	
H = \int\limits_{\mathbf{r}, \mathbf{r}'} \left[ J\left( \mathbf{r} - \mathbf{r}' \right)\mathbf{S}_{\mathbf{r}} \mathbf{S}_{\mathbf{r}'} + {{\mathbf{D}}_{\mathbf{r} \mathbf{r}'}} \cdot \mathbf{S}_{\mathbf{r}} \times \mathbf{S}_{\mathbf{r}'} \right] -  \mathbf{H}_e  \int\limits_{\mathbf{r}}\mathbf{S}_{\mathbf{r}}
\end{equation}

\subsection{Описание сверхрешетки скирмионов}
Как уже было сказано выше, экспериментально наблюдается, что скирмионы в магнетике образуют гексагональную решетку в одной из плоскостей. \cite{muhlbauer, yu} Трехмерные скирмионные структуры можно представить как «многослойку» подобных двумерных решеток, наложенных друг на друга вдоль оси перпендикулярной плоскости с гексагональной решеткой. Мы будем рассматривать только двумерный случай и использовать следующее приближение: вместо рассмотрения плоскости замощенной правильными шестиугольниками мы рассмотрим замощение плоскости дисками. Это означает, что соответствующая гексагональная ячейка аппроксимируется диском равной площади $S_{sk}$ (Рис. \ref{pic:approxHexagone}). Это позволяет избежать сложностей, связанных с рассмотрением нетривиальной границы и использовать полярные координаты для описания конфигурации скирмиона. Различие между шестиугольником и диском на границе не должно сильно влиять на суммарную энергию, поскольку все вектора локальной намагниченности практически коллинеарны на краю скирмиона.

\begin{figure}[h]
\centering
\includegraphics[width=0.5\paperwidth]{images/approx.png}
\caption{Гексагональная ячейка аппроксимируется диском равной площади.}
\label{pic:approxHexagone}
\end{figure}

При этом мы будем выбирать такой радиус $R_\text{opt}$, чтобы энергия образца с решеткой была минимальна,\cite{bogdanov, nagaosaHan} и считать площадь гексагональной ячейки равной этому диску. Длина стороны шестиугольника $\ell$ определяется из условия равенства площадей $\pi R_\text{opt}^{2} = \sqrt{3}\,\ell^{2}/2$. Поскольку полная энергия решетки получается суммированием энергий отдельных скирмионов, то для минимизации её  достаточно минимизировать плотность энергии одного скирмиона на диске.



\subsection{Вывод уравнения на спектр спиновых волн из решеточной модели}
Аналогично работе \cite{aristov1} получим спектральное уравнение на магноны. Введем тензорные обозначения, считая что латинские индексы относятся к спиновым координатам ($a,b,c,d = 1,2,3$), а греческие к пространственным ($\alpha, \beta, \gamma = 1,2$). Тогда, меняя порядок суммирования и вводя вектор $\mathbf{n}=\mathbf{r} - \mathbf{r}'$, получаем из \eqref{eq:ham} выражение в таком виде

\begin{equation}
\label{eq:mainHam}
\int\limits_{{\mathbf{r}},{\mathbf{n}}} {\left\{ { S_{\mathbf{r}}^a\left( {J\left( {\mathbf{n}} \right){\delta _{ab}} + D{\varepsilon _{abc}}{\delta _{\alpha c}}{n^\alpha }} \right) S_{{\mathbf{r}} - {\mathbf{n}}}^b} \right\}}  - s H_e \int\limits_{\mathbf{r}} { \delta _{a3}  S_{\mathbf{r}}^a }
\end{equation}
Здесь мы ввели антисимметричный тензор $\varepsilon_{abc}$ ($\varepsilon_{123} = 1$) и магнитное поле $\mathbf{H}_e$, мы определили в единицах $s$. Заметим, что взаимодействие Дзялошинского-Мория смешивает спиновые и пространственные индексы - это известное свойство спин-орбитального взаимодействия.

Предположим теперь, что основное состояние - это скирмион, и перепишем гамильтониан в таком локальном базисе, что средняя намагниченность направлен	а вдоль оси $z$.
Переход к такому базису ${\mathbf{S}}_{\mathbf{r}} = \hat U\left( \mathbf{r} \right)\tilde { \mathbf{S}}_{\mathbf{r}}$ осуществляется с помошью матрицы $\hat U(\mathbf{r}) = e^{-\alpha \sigma_3}e^{-\beta \sigma_2}e^{-\gamma \sigma_3}$, с соответствующими генераторами $\sigma_2$,$\sigma_3$ группы $SO(3)$ и углами Эйлера $\alpha$, $\beta$, $\gamma$. Угол $\gamma$ не определяется из уравнения на классическую конфигурацию и может быть выбран исходя из соображений того (как показано в \cite{aristov1}), требуем ли мы однозначность и непрерывность матрицы $U$ в нуле ($\gamma = -\alpha$) или на границе ($\gamma = \alpha$). В дальнешем мы положим  $\gamma = -\alpha$.

Теперь проведем градиентное разложение оператора спина. Будем считать, что $|\mathbf{n}|=n \ll |\mathbf{r}|$. Тогда можно воспользоваться разложением, $${S}_{{\mathbf{r}} - {\mathbf{n}}}^b = {S}_{\mathbf{r}}^b - {n^\beta }{\nabla ^\beta }{S}_{\mathbf{r}}^b + \frac{1}{2}{n^\beta }{n^\gamma }{\nabla ^\beta }{\nabla ^\gamma }{S}_{\mathbf{r}}^b$$ где слагаемые с градиентами имеют следующий вид в новом базисе:

\[{\nabla ^\beta }S_{\mathbf{r}}^b = {\nabla ^\beta }\left( {{U^{bd}}{{\tilde S}^d}_{\mathbf{r}}} \right) = \left( {{\nabla ^\beta }{U^{bd}}} \right){{\tilde S}^d}_{\mathbf{r}} + {U^{bd}}\left( {{\nabla ^\beta }{{\tilde S}^d}_{\mathbf{r}}} \right)\]

	
\[{\nabla ^\beta }{\nabla ^\gamma }S_{\mathbf{r}}^b = \left( {{\nabla ^\beta }{\nabla ^\gamma }{U^{bd}}} \right){\tilde S^d}_{\mathbf{r}} + 2\left( {{\nabla ^\beta }{U^{bd}}} \right)\left( {{\nabla ^\gamma }\tilde S_{\mathbf{r}}^d} \right) + {U^{bd}}\left( {{\nabla ^\beta }{\nabla ^\gamma }{{\tilde S}^d}_{\mathbf{r}}} \right)\]
Теперь удобно определить два тензора

\begin{eqnarray}
\label{eq:chi}
\begin{gathered}
  \chi _{1,\alpha }^{ab} = {U^{ca}}{\nabla ^\alpha }{U^{cb}}, \hfill \\
  \chi _{2,\alpha \beta }^{ab} = {U^{ca}}\left( {{\nabla ^\alpha }{\nabla ^\beta }{U^{cb}}} \right) \hfill \\ 
\end{gathered}
\end{eqnarray}
Явные выражениями для матриц $U(\mathbf{r})$ и $\chi_1(\mathbf{r})$, $\chi_2(\mathbf{r})$ известны \cite{paper:aristov}.

В длинноволновом приближении (${\mathbf{qn}} \ll 1$), тогда 
\[J({\mathbf{q}}) = \sum\limits_{\mathbf{n}} {{e^{i{\mathbf{qn}}}}J({\mathbf{n}})}  \simeq J(0) + \frac{C}{2}{q^2}\]
посчитаем теперь внутренние суммы по $\mathbf{n}$. Для обменного слагаемого:

\begin{eqnarray}
\label{eq:integrExch}
\begin{gathered}
  \sum\limits_{\mathbf{n}} {{n^\alpha }J({\mathbf{n}})}  = 0 \hfill \\
  \sum\limits_{\mathbf{n}} {{n^\alpha }{n^\beta }J({\mathbf{n}})}  =  - C{\left. {\frac{{{d^2}J({\mathbf{q}})}}{{d{q^\alpha }d{q^\beta }}}} \right|_{q = 0}} = - C{\delta _{\alpha \beta }} \hfill \\ 
\end{gathered}
\end{eqnarray}
Для вклада от взаимодействия Дзялошинского-Мория:

\begin{eqnarray}
\label{eq:integrDM}
\begin{gathered}
  D\sum\limits_{\mathbf{n}} {{\varepsilon _{abc}}{\delta _{\alpha c}}{n^\alpha }}  = 0 \hfill \\
  D\sum\limits_{\mathbf{n}} {{\varepsilon _{abc}}{\delta _{\alpha c}}{n^\alpha }{n^\beta }}  = D{\varepsilon _{abc}}{\delta _{\beta c}} \hfill \\ 
\end{gathered}
\end{eqnarray}
Используя (\ref{eq:chi}),(\ref{eq:integrExch}) и (\ref{eq:integrDM}) получаем из (\ref{eq:mainHam}) гамильтониан в такой форме
\begin{equation}
\label{eq:hamlitNewBasis}
H \approx {H_{ex}} + {H_{DM}} + {H_Z}
\end{equation}
где $H_{ex}$ вклад от обменного взаимодействия,
\[{H_{ex}} =  - \frac{1}{2}C\int\limits_{\mathbf{r}} {\tilde S_{\mathbf{r}}^a\left( {\chi _{2,\beta \beta }^{ab} + 2\chi _{1,\beta }^{ab}{\nabla ^\beta } + {\delta _{ab}} \nabla ^ 2 } \right)\tilde S_{\mathbf{r}}^b} \]
$H_{DM}$ от Дзялошинского-Мория
\[{H_{DM}} =  - D\int\limits_{\mathbf{r}} {\tilde S_{\mathbf{r}}^a{\varepsilon _{adc}}{\delta _{e\alpha }}{U^{ec}}\left( {\chi _{1,\alpha }^{db} + {\delta _{db}}{\nabla ^\alpha }} \right)\tilde S_{\mathbf{r}}^b} \]
и $H_Z$ от внешнего магнитного поля 
\[{H_Z} =  - s H_e \int\limits_{\mathbf{r}} {{U^{3a}}\tilde S_{\mathbf{r}}^a} \]
Воспользуемся представлением Малеева-Дайсона для спиновых операторов, сохраняющим коммутационные соотношения  ($[\tilde{S}^a,\tilde{S}^b] = i \epsilon_{abc}\tilde{S}^c$):
\begin{equation} 
\begin{aligned} 
\label{eq:boz}
     \tilde{S}^{z}_{j} &=s-a^+_{ j} a_{ j} \,, \quad   
       \tilde{S}^{+}_{j}=\sqrt{2s}a_{ j}  \\
     \tilde{S}^{-}_{j} &=\sqrt{2s}\left( a^{+}_{ j} - \frac{1}{2s}a^+_{ j}a^{+}_{ j}a_{ j} \right)
  \end{aligned}  
 \end{equation} 
здесь $s$ величина спина,  $\tilde S^{\pm} = \tilde S^{x} \pm i \tilde S^{y}$ и $[a_{ j},a^+_{ j}] = 1$.  
С помощью (\ref{eq:boz}) мы получаем из (\ref{eq:hamlitNewBasis}) гамильтониан разложенный по степеням $s$, где слагаемое при $s^2$ имеет вид
\begin{equation}
\label{eq:cls_skx_eng}
{H_c} = \int {d\mathbf{r} \left( {-J\left( 0 \right) + \frac{1}{2}{\text{C}}\left( {\frac{{{{\sin }^2}\beta }}{{2{r^2}}} + {{\left( {\frac{{d\beta }}{{dr}}} \right)}^2}} \right){\text{ + D}}\left( {\frac{{\sin 2\beta }}{{2r}} + \frac{{d\beta }}{{dr}}} \right) - H_e \cos \beta } \right)}
\end{equation} 
Используя уравнение на $\beta$, несложно убедиться (с помощью интегрирования по частям), что слагаемые при степени $s^{3/2}$ сокращаются - это соответствует тому, что в качестве основного состояния взято скирмионное решение, отвечающее локальному минимуму полной энергии \eqref{eq:ham}. 

И, наконец, в порядке по $s$, можно выписать квадратичное по операторам рождения и уничтожения слагаемое в $r$-представлении:

\begin{equation}
\label{eq:HamQuantBose}
{H_q} = \frac{1}{2}\int {d\mathbf{r} \left( {2a_{\mathbf{r}}^\dag \hat F\left( {\mathbf{r}} \right){a_{\mathbf{r}}} + a_{\mathbf{r}}^\dag G^*\left( {\mathbf{r}} \right)a_{\mathbf{r}}^\dag  + {a_{\mathbf{r}}}{G}\left( {\mathbf{r}} \right){a_{\mathbf{r}}}} \right)}
\end{equation}

\[\begin{gathered}
  F({L_z}) \equiv C\left( { - \nabla ^ 2  + \frac{{1 + 3\cos 2\beta }}{{4{r^2}}} - \frac{{2\cos \beta }}{{{r^2}}}{{\text{L}}_{\text{z}}} - \frac{1}{2}{{\left( {\frac{{d\beta }}{{dr}}} \right)}^2}} \right) \hfill \\
  \,\,\,\,\,\,\,\,\,\,\,\,\,\, + D\left( { - \frac{{3\sin 2\beta }}{{2r}} + \frac{{2\sin \beta }}{r}{{\text{L}}_{\text{z}}} - \frac{{d\beta }}{{dr}}} \right) + H_e \cos \beta  \hfill \\ 
\end{gathered} \]

\[G \equiv \frac{C}{2}\left( { - \frac{{{{\sin }^2}\beta }}{{{r^2}}} + {{\left( {\frac{{d\beta }}{{dr}}} \right)}^2}} \right) + {\text{D}}\left( {\frac{{d\beta }}{{dr}} - \frac{{\sin 2\beta }}{{2r}}} \right)\]
где определены операторы ${{\text{L}}_{\text{z}}} \equiv  - i\frac{\partial }{{\partial \varphi }}$
и 
$\nabla ^ 2  \equiv \frac{1}{r}\frac{\partial }{{\partial r}}\left( {r\frac{\partial }{{\partial r}}} \right) - \frac{{{\text{L}}_z^2}}{{{r^2}}}$ 


\subsection{Единицы измерения и функционал плотности классической энергии скирмиона на диске }
Мы получили выражение для классической энергии скирмиона \eqref{eq:cls_skx_eng} из которого следует исключить константу $-J(0)$ и ввести более удобные единицы измерения, измеряя расстояние в единицах $C/D$ и энергию в единицах $E_{0} = \mu D^{2}/ C$. Тогда в \eqref{eq:cls_skx_eng} остается единственный безразмерный параметр $b$ равный $H_e C/(\mu D^2)$. Во всех дальнейших рассуждениях, связанных с численными расчётами, мы будем полагать $b=0.6$, если значение не оговорено особо.  Классический вклад в энергию имеет большой префактор в энергию $\mu \sim s$ и определяется статической конфигурацией $\mathbf{S}(\mathbf{r}) $, минимизирующей энергию $\langle H\rangle$.  Легко убедиться тогда, что вклад от обменного взаимодействия в подынтегральном выражении в энергию принимает вид:


\begin{equation}
\label{eq:ExchClass}
 {\left( {\nabla {\mathbf{S}}} \right)^2} = {\left( {\frac{{\partial \beta }}{{\partial r}}} \right)^2} + \frac{1}{{{r^2}}}{\left( {\frac{{\partial \beta }}{{\partial \varphi }}} \right)^2} + {\sin ^2}\beta \left( {{{\left( {\frac{{\partial \alpha }}{{\partial r}}} \right)}^2} + \frac{1}{{{r^2}}}{{\left( {\frac{{\partial \alpha }}{{\partial \varphi }}} \right)}^2}} \right)
\end{equation}
от Дзялошинского-Мория:

\begin{equation}
\label{eq:DMclass}
\begin{gathered}
  {\mathbf{S}} \cdot \left[ {\nabla  \times {\mathbf{S}}} \right] = \left( {\frac{{\partial \alpha }}{{\partial r}}\cos \left( {\alpha  - \varphi } \right) + \frac{1}{r}\frac{{\partial \alpha }}{{\partial \varphi }}\sin \left( {\alpha  - \varphi } \right)} \right)\sin \beta \cos \beta  \hfill \\
  \,\,\,\,\,\,\,\,\,\,\,\,\,\,\,\,\,\,\,\,\,\,\, + \frac{{\partial \beta }}{{\partial r}}\sin \left( {\alpha  - \varphi } \right) - \frac{1}{r}\frac{{\partial \beta }}{{\partial \varphi }}\cos \left( {\alpha  - \varphi } \right) \hfill \\ 
\end{gathered}
\end{equation}
и, наконец, от магнитного поля

\begin{equation}
\label{eq:ExtClass}
{\mathbf{H}_e} \cdot {\mathbf{S}} =  b \cos \, \beta 
\end{equation}
Собирая вклады, мы можем получить общее выражение для классического гамильтониана в полярных координатах. 

Для того, чтобы получить уравнение Эйлера-Лагранжа на скирмионную конфигурацию, необходимо задать граничные условия на функции $\alpha$ и $\beta$. С этой целью необходимо обратиться к топологическим свойствам отображений сферы на сферу, что будет сделано в следующем разделе. Однако, сразу отметим, что мы будем интересоваться только скирмионом с $Q=-1$ и $\gamma_0=\pi/2$. Если мы захотим минимизировать плотность энергии скирмиона на диске с радиусом $R_0$, аналогично работе \cite{bogdanov}, то нам нужно будет записать энергию на диске в безразмерных единицах и поделить её на площадь диска. При этом удобно будет вычесть постоянную энергию основного состояния ферромагнетика $b$ (так, чтобы при $\beta \left( r \right) \equiv 0$ энергия была равна нулю). Соответствующий функционал плотности энергии принимает вид

\begin{equation}
\label{eq:density}
{\rho _c} = \frac{2}{{R_0^2}}\int\limits_0^{{R_0}} {dr\left( {\frac{{{{\sin }^2}\beta }}{r} + r{{\left( {\frac{{d\beta }}{{dr}}} \right)}^2} + r\frac{{d\beta }}{{dr}} + \frac{{\sin 2\beta }}{2} - br\left( {\cos \beta  - 1} \right)} \right)}
\end{equation}

\subsection{Топологический заряд}

Из \cite{book:rajaraman} известно, что отображения вида $\mathcal{S}^2 \rightarrow \mathcal{S}^2$ могут быть подразделены на гомотопические секторы. Причем множество таких секторов, или классов, счетно и может быть охарактеризовано набором целых чисел $Q$. Это число, выделяющее ту или иную конфигурацию, называют топологическим зарядом. В нашем случае $Q$ определяет число обходов внутренней сферы при вращении координатного пространства $\mathbb{R}^2 \bigcup \{\infty\}$, сжатого до $\mathcal{S}^2$ за счет условия нормировки.

Таким образом, закрученная структура скирмиона определяется топологическим зарядом \cite{rajaraman}:

\begin{equation}
\label{eq:topChargeMain}
Q \equiv \frac{1}{{4\pi }}\int {d{\mathbf{r}}\left( {{\mathbf{S}}\left[ {\frac{{\partial {\mathbf{S}}}}{{\partial x}} \times \frac{{\partial {\mathbf{S}}}}{{\partial y}}} \right]} \right)}
\end{equation}
Если воспользоваться параметризацией для единичного вектора $\mathbf{S}$:
\begin{equation}
\label{eq:parametr}
\mathbf{S} = \left( {\begin{array}{*{20}{c}}
{\cos \alpha \sin \beta }\\
{\sin \alpha \sin \beta }\\
{\cos \beta }
\end{array}} \right)
\end{equation}
и положить $\alpha = \alpha(\phi)$ и $\beta = \beta (r)$, то, считая скирмион заданным на диске радиуса $R_0$,  выражение \eqref{eq:topChargeMain} преобразуется к
\begin{equation}
\label{eq:topChargeSimplify}
\left. {Q = \frac{1}{{4\pi }}\int\limits_0^{{R_0}} {dr\int\limits_0^{2\pi } {d\varphi } } \frac{{d\alpha }}{{d\varphi }}\frac{{d\beta }}{{dr}}\sin \beta  =  - \frac{1}{{2\pi }}\alpha \left( \varphi  \right)} \right|_{\varphi  = 0}^{\varphi  = 2\pi }\left. {\frac{1}{2}\cos \beta \left( r \right)} \right|_{r = 0}^{r = {R_0}}
\end{equation}
Отсюда можно заключить, что для задания нетривиальной топологической конфигурации ($Q \neq0$) можно взять следующие граничные условия:

\begin{equation}
\label{eq:edgeCond}
{\left\{ {\begin{array}{*{20}{c}}
  {\alpha \left( \varphi  \right) = {W_0}\varphi  + {\gamma _0}} \\ 
  {\beta (0) = \pi } \\ 
  {\beta ({R_0}) = 0} 
\end{array}} \right.}
\end{equation}
Здесь $W_0$ это целое число, а $\gamma _0$ произвольно. При таких условиях мы получаем $Q=-W_0$. Известно, что энергия скирмиона тем выше, чем больше $|Q|$ \cite{rajaraman}, поэтому будем рассматривать конфигурации с $Q=\pm 1$. В данной работе анализируется "baby skyrmion" с $Q=-1$, и следовательно $W_0 = 1$.

В отсутствии стабилизирующих взаимодействий, энергия не зависит от $\gamma_0$. В нашем случае это не так. В самом деле, используя \eqref{eq:DMclass} и \eqref{eq:edgeCond} получаем

\[\mathbf{D} \, {\mathbf{S}} \cdot \left[ {\nabla  \times {\mathbf{S}}} \right] = D \, \sin \left( {\left( {{W_0} - 1} \right) \varphi  + {\gamma _0}} \right)\,\left( {\frac{{d\beta }}{{dr}} + {W_0}\frac{{\sin 2\beta }}{{2r}}} \right)\]
Мы хотим минимизировать общую энергию, значит интеграл от этого слагаемого должен быть отрицательным. Поскольку $\int {d{r}} \left( {\frac{{d\beta }}{{dr}} + {W_0}\frac{{\sin 2\beta }}{{2r}}} \right) < 0$ для скирмионной конфигурации  при $W_0=1$, то множитель перед интегралом должен быть положительным, кроме того, этот вклад в энергию будет максимальным при $\gamma_0 = \pi/2$ если $D>0$. В этом выражении восстановлена константа $D$, чтобы показать, что направление закрутки скирмиона $\gamma_0 = \pm \pi/2$ определяется знаком $D$.


\subsection{Асимптотики профиля скирмиона $\beta$ и плотность классической энергии }

Уравнение Эйлера-Лагранжа для \eqref{eq:density} с граничными условиями \eqref{eq:edgeCond} на $\beta$ принимает вид
\begin{equation}
\label{eq:eulerLagr}
\frac{{{d^2}\beta }}{{d{r^2}}} + \frac{1}{r}\frac{{d\beta }}{{dr}} - \frac{{\sin \beta \cos \beta }}{{{r^2}}} + \frac{{2{{\sin }^2}\beta }}{r} - b\sin \beta  = 0
\end{equation}
Данное уравнение выглядит довольно громоздко и решить его аналитически не представляется возможным. Однако оно допускает несложный численный анализ и позволяет получить профиль скирмиона численно для конкретного радиуса диска. Подставляя этот профиль в (\ref{eq:density}), и вычисляя $\rho_c$, получается зависимость плотности энергии от радиуса диска, изображенная на Рис. \ref{pic:plotDensity}.

Асимптотики решения уравнения \eqref{eq:eulerLagr} имеют вид:
\begin{eqnarray*}
\label{eq:asympt_beta}
\beta(r) &\sim \pi - c_1 \sqrt{b} r ,\qquad &r \rightarrow 0 \\
\beta(r) &\sim \frac{c_2}{b^{\frac{1}{4}}\sqrt{r}} e^{-\sqrt{b} r} ,\qquad & r \rightarrow \infty
\end{eqnarray*}
с некоторыми коэффициентами $c_1>0$ и $c_2>0$. Из асимптотики $\beta$ можно заключить, что величина $\frac{1}{\sqrt{b}}$ определяет характерный радиус скирмиона.

\begin{figure}[t]
\centering\includegraphics[width=0.65\paperwidth]{images/plotDensity.pdf}
\caption{Зависимость $\rho _c$ от радиуса диска при различных магнитных полях (параметр $b$). При $b\gtrsim 0.8$ не наблюдается минимум и энергия скирмиона положительна.}
\label{pic:plotDensity}
\end{figure}
Из (рис. \ref{pic:plotDensity}) видно, что существует оптимальный радиус $R_{\text{opt}}$ минимизирующий плотность в определенном интервале величин внешнего магнитного поля. При достаточно большом поле (при $b > b_{crit} \approx 0.8$) не наблюдается минимум энергии скирмиона и скирмионное состояние метастабильно.

\subsection{ Сравнение плотностей энергий скирмионной решетки и спирали }

Стоит отметить, что одиночная спираль тоже является хорошим кандидатом для истинного основного состояния системы описываемой уравнением \eqref{eq:density}. Плотность энергии для конической спирали в 3D случае без ферромагнитного вклада $-B$, дается выражением \cite{Maleyev2006}
 \begin{equation}
\begin{aligned}
\rho_{con} & = - \frac{D^{2}}{2C} (b-1)^{2} \,, \quad  0<b<1   \,, \\
 &= 0   \,, \quad  b \ge1 \,,  \\
\end{aligned}
\label{helix_energy}
\end{equation}
В 3D случае ось конуса конической спирали $q=D/C$ параллельна направлению поля, в 2D мы имеем следующее выражение для плотности энергии
 \begin{equation}
\begin{aligned}
\rho_{hel} & = \frac{D^{2}}{C} (-1/2+b) \,, \quad  0<b<1/2   \,, \\
 &= 0   \,, \quad  b \ge1/2 \,,  \\
\end{aligned}
\label{helix_energy2D}
\end{equation}
Для небольших полей ось конуса лежит в плоскости и его апертура равна $ \pi $ (геликоида); для более сильных полей ($ b > 1/2 $) спираль имеет большую энергию, чем конфигурация с равномерным спином.

На Рис.\ref{fig:Sk_v_helix} мы сравниваем плотности энергий описываемых уравнениями \eqref{helix_energy}, \eqref{helix_energy2D} с плотностью энергии определяемой скирмионной конфигурацией уравнения \eqref{eq:density}.


\begin{figure}[t]
\centering	
\includegraphics[width=0.85\columnwidth]{images/Sk_v_helix.pdf}
\caption{Классическая плотность энергии скирмиона (пунктирная линия, уравнение \eqref{eq:density}) показана в сравнении с энергией конической спирали в 3D (сплошная линия, уравнение \eqref{helix_energy}) и геликоидой в 2D (пунктирная линия, уравнение \eqref{helix_energy2D}). Кривые нормированы множителем $s$.}
\label{fig:Sk_v_helix}
\end{figure}

Отсюда видно, что в 2D случае скирмионная конфигурация выигрывает по энергии для полей  $0.17 \lesssim b \lesssim 0.8$, в соответствии с \cite{Rybakov2015}.
Энергия конической спирали имеет меньшее значение, чем у скирмионной конфигурации во всем диапазоне полей, $b \in (0,1)$. Это означает, что конфигурацию скирмиона в 3D следует рассматривать как метастабильную. Хотя этот факт не мешает нам определить спектр магнонов ниже, он может поставить под сомнение такую процедуру. Отметим здесь, что обсуждаемые квантовые поправки к основному состоянию понижают энергию конфигурации скирмионов и, в конечном итоге, могут сделать ее предпочтительной для системы. Подтверждение последнего утверждения требует, однако, расчета квантовой поправки к состоянию спирали, что выходит за рамки настоящего исследования.


\pagebreak
\section{ Уравнение на спектр спиновых волн в базисе функций Ландау }

\subsection{ Калибровочный потенциал }

Гамильтониан (\ref{eq:HamQuantBose}) удобно представить в следующем матричном виде, используя введенные ранее безразмерные единицы измерения:
\begin{equation}
\label{eq:HamQuantBoseDimensionless}
\hat {\cal H}_{q} =\begin{pmatrix} F(\hat  L_{z}) &g \\ g & F(  - \hat  L_{z} ) \end{pmatrix}
\end{equation}
где используются следующие определения
$$
\begin{aligned} 
F  (\hat  L_{z} )&\equiv   - \nabla^{2}  + \frac{1 + 3\cos 2\beta }{4r^2} - \frac{2\cos \beta }{r^2}\hat{L}_{z} - \frac{1}{2}\left( \frac{d\beta }{dr} \right)^2  \\
& +  \left\{  - \frac{3\sin 2\beta }{2r} + \frac{2\sin \beta }{r}\hat{L}_{z}  - \frac{d\beta }{dr} \right\} + b\cos \beta   \\ 
g &\equiv \frac{1}{2}\left(  - \frac{\sin ^2\beta }{r^2} + \left( \frac{d\beta }{dr} \right)^2 \right) 
+ \left\{ \frac{d\beta }{dr} - \frac{\sin 2\beta }{2r} \right\}
\end{aligned}  
$$

Данное выражение было выведено для одного скирмиона, расположенного в начале координат, на диске оптимального радиуса $R_\text{opt}$. Мысленно заполняя плоскость одинаковыми дисками, мы можем предположить, что полный гамильтониан \ref{eq:HamQuantBose} скирмионного кристалла также может быть аппроксимирован на всю плоскость. При этом оператор потенциальной энергии должен быть продолжен периодическим образом на всю решётку. Для того, чтобы проиллюстрировать эту идею, удобно ввести калибровочный потенциал:


\begin{equation}
(- i \nabla - \mathbf{A} ) ^ 2 = - \nabla ^ 2 + 2 i \mathbf{A} \cdot \nabla + i \text{div} \mathbf{A}  + \mathbf{A}^2
\end{equation}

$$
\mathbf{A} = A(r) \mathbf{e}_\phi ,  \qquad  \hat{L}_{z} = - i r \nabla_\phi
$$
где $A(r) = \frac{ \cos\beta } { r } - \sin \beta$ и плотность потока принимает вид:
$$
{\cal B}  = \frac 1 r \nabla_r \left(r A \right) = 
% - \frac 1 r \left( \left( \sin \beta + r \cos \beta \right) \beta' + \sin \beta \right)
r^{-1}  \nabla_r \left(\cos\beta   -  r  \sin \beta \right)  
$$
Выражение \eqref{eq:HamQuantBoseDimensionless} при этом переписывается в следующем виде
\begin{equation}
\hat {\cal H}_{q} =\begin{pmatrix} (- i \nabla - \mathbf{A} ) ^ 2 + h_1 & g \\ g & ( i \nabla - \mathbf{A} ) ^ 2 + h_1 \end{pmatrix}
\end{equation}
где
\begin{equation}
h_1(r) \equiv  -\frac{\sin^2 \beta }{2r^2} - \frac{1}{2}\left( \frac{d\beta }{dr} \right)^2 +  \left\{  - \frac{\sin 2\beta }{2r}  - \frac{d\beta }{dr} \right\} + b\cos \beta - \sin ^ 2 \beta
\end{equation}
Теперь если мы рассмотрим калибровочное преобразование для введеного поля 
$$
\mathbf{A} \rightarrow \mathbf{A} + \nabla f(\mathbf{r})
$$
$$
\psi \rightarrow \psi e ^ {- i f(\mathbf{r})}
$$
с некоторой функцией  $f = P \varphi$, где константа $P \in \mathbb{N}$. Потенциал преобразуется к следующему виду:
$$
\mathbf{A'}= \left(\frac {P + \cos{\beta}} {r} - \sin{\beta} \right) \mathbf{e}_\varphi
$$
которое преобразует волновую функцию к
$$
\psi' = \psi e^{-i P \varphi} 
$$
И тогда гамильтониан принимает следующий вид в данной калибровке
\begin{equation}
\label{eq:init_ham_landau}
\hat {\cal H}_{q} \vec{\Psi} = \tau_z E \vec{\Psi}
\end{equation}
$$
\hat {\cal H}_{q} =\begin{pmatrix} (- i \nabla - \mathbf{A} ) ^ 2 + h_1 (r) &  g (r) e^{2 i P \varphi} \\  g (r) e^{-2 i P \varphi}   & ( i \nabla - \mathbf{A} ) ^ 2 + h_1 (r) \end{pmatrix}, \vec{\Psi} = \begin{pmatrix}  \psi_1 \\ \psi_2 \end{pmatrix}
$$
Замечу здесь, что выбор $P=1$ удобен тем, что $A' \sim r^2, r \rightarrow 0$ и вектор-потенциал является непрерывным в нуле, это видно из асимптотик \eqref{eq:asympt_beta} функции $\beta$. В дальнейшем будем считать, что мы выбрали именно эту калибровку. В итоге, получается обобщенное уравнение на собственные функции \eqref{eq:init_ham_landau}, где условие нормировки на волновую функцию выглядит следующим образом:
$$
\vec{\Psi}^\dagger \tau_z \vec{\Psi} =|\psi_1|^2-|\psi_2|^2=1
$$


\subsection{ Выделение затравочной и основной части в гамильтониане }

Мы заметили ранее, что поле $\cal B$ направлено вдоль оси $\mathbf{e}_z$ и зависимость от $r$ дается выражением  ${\cal B} = r^{-1} d(r A(r))/dr$. Данное поле ещё обладает тем свойством, что поток его через единичную скирмионную ячейку оказывается равен $ -4\pi $ и пропорционален топологическому заряду  $Q$, как видно в \eqref{eq:topChargeSimplify}:

\begin{equation}
\Phi = \int \limits_\Omega {\cal B} \mathbf{e}_z d\mathbf{S} = \int_\phi d \phi \int_r d r \nabla_r \left(\cos\beta   -  r  \sin \beta \right)  = 4 \pi Q
\end{equation}

Теперь, замечая это свойство потока мы разобъем его на две части, одну, отвечающаю однородному полю и вторую с нулевым потоком, для этого вычтем поток однородного поля ${\cal B}_{0} = 4/R_{0}^{2}$  из полной плотности потока  $\cal B$, и оставшаяся часть ${\cal B}_{1} = {\cal B} - {\cal B}_{0}$ очевидно будет обладать нулевым потоком.
 
Эта операция на самом деле соответствует тому, что калибровочный потенциал в гамильтониане \eqref{eq:init_ham_landau} разбивается на две части $A_0,A_1$:
\begin{equation}
\begin{aligned}
\mathbf{A}  & = \left(A_0 + A_1  \right) \mathbf{e}_\phi, \quad A_0 =   \frac 1 2 r {\cal B}_{0} ,    \\
\quad A_1 & = \frac{1+  \cos\beta } { r } - \sin \beta - \frac 1 2 r {\cal B}_{0}     \\
\end{aligned}
\label{eq:devide_2_hamilte}
\end{equation}
Заметим, по построению $A_{1}(0) = A_{1}(R_{0}) = 0 $ и  видно, что $(\pm i \nabla - \mathbf{A} ) ^ 2$  в гамильтониане \eqref{eq:init_ham_landau} разбиваются на
\begin{equation}
\begin{aligned}
(\pm i \nabla - \mathbf{A} ) ^ 2 & = (\pm i \nabla - \mathbf{A}_0 ) ^ 2 - 
2 \mathbf{A}_1  (\pm i \nabla - \mathbf{A}_0 )+ \mathbf{A}_1^2   \\
\end{aligned}
\end{equation}
поскольку $ \mbox{div } \mathbf{A}_1=0$. 
При этом  можно выделить слагаемое гамильтониана \eqref{eq:init_ham_landau} как основную часть
$$
\begin{pmatrix} ( -i \nabla - \mathbf{A}_0 ) ^ 2 & 0 \\ 0 & ( i \nabla - \mathbf{A}_0 ) ^ 2 \end{pmatrix}
$$
и дополнительную, затравочную
$$
\begin{pmatrix} \mathbf{A}_1  (i \nabla - \mathbf{A}_0 ) + u(r) & g (r) e^{2 i  \varphi} \\ g (r) e^{-2 i  \varphi} & \mathbf{A}_1  (-i \nabla - \mathbf{A}_0 ) + u(r) \end{pmatrix}
$$ 
где $u(r) \equiv  h_1(r) + \mathbf{A}_1^2$. 

Решение для основной части хорошо известно и может быть выписано в базисе:
\begin{equation}
\label{eq:basis_func_landau}
\ket{n_r, m, \alpha} = \frac{e^{(-1)^{\alpha+1}  i m \phi}} {\sqrt{2 \pi}} \Upsilon_{m,n_r} (r)
\end{equation}
как
$$
\bra{n_r', m', \beta} \hat h (r)  \ket{n_r, m, \alpha} = 2 {\cal B}_{0} \left( n_r + \frac{|m| - m + 1}{2} \right) \delta_{\alpha \beta} \delta_{n_r n'_r} \delta_{m m'} 
$$
\noindent где  
\begin{equation}
\begin{aligned}
\Upsilon_{m, n_r} (r) & = {\cal B}_{0}^{\frac{1+|m|}{2}} \sqrt{\frac{(|m| + n_r)!}{2^{|m|}n_r!(|m|!)^2}} e^{-\frac{{\cal B}_{0} r^2}{4}} r^{|m|} F\left(-n_r,|m| + 1,\frac{{\cal B}_{0} r^2 }{2} \right) 
\end{aligned}
%\label{}
\end{equation}
\noindent и $F$  это гипергеометрическая функция с условием ортогональности 
$$ \int\limits_0^\infty \Upsilon_{m, n_r}^2 (r) r dr = 1$$
и $n_r \in \mathbb{Z}_+$, $m \in \mathbb{Z}$.

Оставшиеся части гамильтониана \eqref{eq:init_ham_landau}, $\mathbf{A}_1^2$, $g(r)$, $u(r)$ и особое слагаемое 
$\mathbf{A}_1  (\pm i \nabla - \mathbf{A}_0 )$ могут быть рассмотрены как периодическое возмущение основного состояния.


\subsection{Слагаемые периодической части возмущения основного гамильтониана}

Уравнение Шредингера в базисе \eqref{eq:basis_func_landau} выписывается в матричном виде как:
\begin{equation}
\label{eq:shred_final}
\mel{m', n'_r, \alpha}{\hat h^{\alpha \beta}(\mathbf{r}) + \hat v^{\alpha \beta}(\mathbf{r}) + \hat u^{\alpha \beta}(\mathbf{r}) + \hat g^{\alpha \beta} (\mathbf{r})) }{n_r,m, \beta} = {\tau_z^{\alpha \beta} \delta_{n_r n'_r} \delta_{m m'} E_{n_r m} }
\end{equation}

\noindent где:
\begin{equation}
\label{eq:definitions}
\begin{aligned}
\hat h^{\alpha \beta}(\mathbf{r})   &= \left( (-1)^\alpha i \nabla - \mathbf{A}_0 \right) ^ 2 \delta_{\alpha \beta} \\
\hat v^{\alpha \beta}(\mathbf{r})   &= - 2 \mathbf{A}_1 \left( (-1)^{\alpha} i \nabla - \mathbf{A}_0 \right) \delta_{\alpha \beta} \\
\hat u^{\alpha \beta}(\mathbf{r})   &= u(\mathbf{r})  \delta_{\alpha \beta}\\
\hat g^{\alpha \beta} (\mathbf{r}) &=   g(\mathbf{r})  e^{ (-1)^{\beta} 2  i   \varphi} \tau^{\alpha \beta}_x 
\end{aligned}
\end{equation}

\noindent Выясним точный вид для каждого слагаемого. 

\subsubsection{Невозмущенная часть: $\hat h$}
Выражение для невозмущенной части мы уже знаем, но выпишем его для удобства ещё раз:
\begin{equation}
\mel{n'_r, m', \alpha} {\left((-1)^\alpha i \nabla - \mathbf{A}_0 \right) ^ 2 \delta_{\alpha \beta}} {n_r, m, \beta} = 
2 {\cal B}_{0} \left( n_r + \frac{|m| - m + 1}{2} \right) \delta_{\alpha \beta} \delta_{n_r n'_r} \delta_{m m'} 
\end{equation}


\subsubsection{Возвмущенная часть: специальное слагаемое $\hat v$}
Для  $\hat v$ части \eqref{eq:shred_final} удобно воспользоваться следующими соотношениями:
\begin{equation}
\begin{aligned}
-2 \mathbf{A}_1  (-i \nabla - \mathbf{A}_0 )& = \sqrt{2} \left( A_{1} (\mathbf{r}) e^{ i \varphi  } d^{\dagger} +  A_{1} (\mathbf{r}) e^{-i \varphi  } d \right), \quad \alpha = 1 \\
-2 \mathbf{A}_1  (i \nabla - \mathbf{A}_0 ) & = \sqrt{2} \left( A_{1} (\mathbf{r}) e^{  i \varphi  } c^{} +  A_{1} (\mathbf{r}) e^{-i \varphi  } c^{\dagger} \right) , \quad \alpha = 2
\end{aligned}
\end{equation}
где 
\begin{equation}
\begin{aligned}
A^{\pm}_{1}  & = A_{1}^{x} \pm i A_{1}^{y} = \pm i A_1 (r) e^{ \pm i \varphi  }    \\
d  & =  \tfrac{1}{\sqrt{2}} [  \left(\partial_{x} + \tfrac 12 {\cal B} x  \right) + i   \left(  \partial_{y} + \tfrac 12 {\cal B} y\right) ] \\
d^{\dagger}  & =  \tfrac{1}{\sqrt{2}} [ \left( - \partial_{x} + \tfrac 12 {\cal B} x \right) + i   \left( \partial_{y} - \tfrac 12 {\cal B} y  \right) ] \\
c  & =  \tfrac{1}{\sqrt{2}} [  \left(\partial_{x} + \tfrac 12 {\cal B} x  \right) - i   \left(  \partial_{y} + \tfrac 12 {\cal B} y\right) ] \\
c^{\dagger}  & =  \tfrac{1}{\sqrt{2}} [ \left( - \partial_{x} + \tfrac 12 {\cal B} x \right) + i   \left( - \partial_{y} + \tfrac 12 {\cal B} y  \right) ]
\end{aligned}
%\label{}
\end{equation} 
операторы $c,d$ и $c^{\dagger},d^{\dagger} $ являются операторами понижения и повышения уровня Ландау и имею следующий вид в полярных координатах:
\begin{equation}
\begin{aligned}
d &= \frac{e^{i \varphi}} {\sqrt{2}} \left(  \frac {\partial }{\partial r} + i \frac{1}{r}  \frac {\partial }{\partial \varphi} + \frac 1 2   r \cal B  \right) \\
d^{\dagger} &= \frac{e^{ -i \varphi}} {\sqrt{2}} \left(  - \frac {\partial }{\partial r} + i \frac{1}{r}  \frac {\partial }{\partial \varphi} + \frac 1 2    r \cal B \right) \\
c &= \frac{e^{- i \varphi}} {\sqrt{2}} \left(  \frac {\partial }{\partial r} - i \frac{1}{r}  \frac {\partial }{\partial \varphi} + \frac 1 2   r \cal B  \right) \\
c^{\dagger} &= \frac{e^{ i \varphi}} {\sqrt{2}} \left(  - \frac {\partial }{\partial r} - i \frac{1}{r}  \frac {\partial }{\partial \varphi} + \frac 1 2    r \cal B \right) \\
\end{aligned}
\end{equation}
Тогда легко проверить, что:
\begin{equation}
\begin{aligned}
&\mel{n_r',m',\alpha} {\hat v^{\alpha \beta}(\mathbf{r}) }{n_r, m,\beta} =
 \delta_{\alpha \beta} \sqrt{ {2\cal{B}} } \Big( \\
&\bra{n_r',m',\alpha} (-1)^{\Theta(m-1)}  \sqrt{n_r + 1 + \frac{|m| - m}{2}  }  A_{1} (\mathbf{r}) e^{ -(-1)^{\alpha} i \varphi}   \ket{n_r+\Theta(m-1), m-1,\beta}  \\
&+ \bra{n_r',m',\alpha} (-1)^{\Theta(m)} \sqrt{  n_r + \frac{|m| - m}{2}  }  A_{1} (\mathbf{r}) e^{(-1)^\alpha i \varphi} \ket{n_r-\Theta(m), m +1 ,\beta}   \Big) \\
\end{aligned}
\end{equation}

\noindent где $\Theta$ - функция Хевисайда
$$
\Theta(n)=\begin{cases} 0, & n < 0, \\ 1, & n \ge 0, \end{cases},
$$
и
\begin{equation}
\begin{aligned}
d \ket{0, m'} &= 0 \\
c \ket{0, -m'} &= 0 \\
\end{aligned}
\end{equation}
\noindent и $m \ge 0$.


\subsubsection{Возвмущенная часть: диагональное слагаемое $\hat u$}
Для  $\hat v$ части \eqref{eq:shred_final} мы получаем уравнение в общем виде:
\begin{equation}
\mel{m', n'_r,\alpha} {u(\mathbf{r})  \delta_{\alpha \beta}}{n_r,m, \beta} = i^{m_{\alpha \beta} }  \delta_{\alpha \beta} \sum\limits_{\mathbf{k} }      e^{- i m_{\alpha \beta} \varphi_{\mathbf{k}}} u(\mathbf{k})\mel{\Upsilon_{m', n'_r}} {J_{m_{\alpha \beta}} (kr) }{\Upsilon_{m, n_r}}
\end{equation}

\noindent где $ m_{\alpha \beta} = (-1)^{\beta}m + (-1)^{\alpha+1} m' $
\begin{equation}
u (\mathbf{k}) \approx \frac{2}{R_0^2}  \int\limits_{0}^{R_0}  J_0 (k r ) u(r) r dr 
\end{equation} 
и мы знаем как функция $u$ представлена на диске:
\begin{equation}
u(r) = -\frac{\sin^2 \beta }{2r^2} - \frac{1}{2}\left( \frac{d\beta }{dr} \right)^2 +  \left\{  - \frac{\sin 2\beta }{2r}  - \frac{d\beta }{dr} \right\} + b\cos \beta - \sin ^ 2 \beta + A^2_1(r)
\end{equation}

\subsubsection{Возвмущенная часть: внедиагональное слагаемое $\hat g$}

\begin{equation}
\begin{aligned}
&\mel{m', n'_r, \alpha} {g(\mathbf{r})   e^{ (-1)^{\beta} 2  i   \varphi} \tau^{\alpha \beta}_x} {n_r,m,\beta} = \\ 
&i^{m_{\alpha \beta}} \tau^{\alpha \beta}_x \sum\limits_{\mathbf{k} }      e^{-i m_{\alpha \beta} \varphi_{\mathbf{k}}}  g(\mathbf{k})\mel{\Upsilon_{m', n'_r}} {J_{m_{\alpha \beta}} (kr) }{\Upsilon_{m, n_r}}
\end{aligned}
\end{equation}
\noindent где 
\begin{equation}
g (\mathbf{k}) \approx \frac{1}{\pi R_0^2}  \int  g(\mathbf{r})   e^{ (-1)^{\beta} 2  i   \varphi} e^{-i \mathbf{k} \mathbf{r}}  d\mathbf{r} = - \frac{2}{R_0^2}  \int\limits_{0}^{R_0}  J_2 (k r ) g(r) r dr 
\end{equation} 
и функция $g$ на диске имеет вид:
\begin{equation}
g(r) = \frac{1}{2}\left(  - \frac{\sin ^2\beta }{r^2} + \left( \frac{d\beta }{dr} \right)^2 \right) 
+ \left\{ \frac{d\beta }{dr} - \frac{\sin 2\beta }{2r} \right\}
\end{equation}

\subsection{ Описание численного метода  }


Обобщенное уравнение \eqref{eq:init_ham_landau} может быть переписано в стандартной форме, если ввести матрицу
$$
\tau_z^{1/2} = 
	\begin{pmatrix} 	
		1 &  0 \\
		0 &  i 
	\end{pmatrix}
$$
которая обладает свойствами $(\tau_z^{1/2})^2 = \tau_z$ и $\tau_z^{-1/2}=(\tau_z^{1/2})^{-1}=(\tau_z^{1/2})^\dagger$.
Используя эту матрицу и снова переопределяя вектор 
$$
\vec{\Psi}'=\tau_z^{1/2} \vec{\Psi}
$$
и модифицируя уравнение
$$
\tau_z^{-1/2} \hat {\cal H}_{r} \tau_z^{-1/2} \vec{\Psi}' = E \vec{\Psi}'
$$
мы приходим окончательно к стандартному уравнению на собственные числа:
$$
\hat {\cal H'}_{r} = \tau_z^{-1/2} \hat {\cal H}_{r} \tau_z^{-1/2} = \begin{pmatrix} 
(- i \nabla - \mathbf{A} ) ^ 2 + h_1 (r) &  -i g (r) e^{2 i C \varphi} \\  
-i g (r) e^{-2 i C \varphi}   & - ( i \nabla - \mathbf{A} ) ^ 2 - h_1 (r) 
\end{pmatrix}
$$
$$
\hat {\cal H'}_{r} \vec{\Psi}' = E \vec{\Psi}'
$$
Решая данное уравнение, мы можем получить решение исходного уравнения \eqref{eq:main_eq_landau}, произведя замену
$$
\vec{\Psi} = \tau_z^{-1/2} \vec{\Psi}'
$$


\pagebreak
\section{ Анализ основных результатов }


\specialsection{Выводы}


\pagebreak
\specialsection{Заключение}
% Библиография в cpsconf стиле
% Аргумент {1} ниже включает переопределенный стиль с выравниванием слева
\pagebreak
\begin{thebibliography}{1}
\bibitem{ahiezer} Ахиезер А.И. et al. \flqq Спиновые волны\frqq. Наука, 1967

\bibitem{aristov1} Aristov, D.N. et al. \flqq Magnon spectrum in ferromagnets with a skyrmion\frqq. JETP Lett. 102, 2015, pp. 511

\bibitem{belavin} Белавин, А. А. et al. \flqq Метастабильные состояния двумерного изотропного ферромагнетика\frqq. Письма в ЖЭТФ №10 v. 22, 1975, pp. 503-506

\bibitem{bogdanov} Bogdanov, A., et al. \flqq Thermodynamically stable magnetic vortex states in magnetic crystals\frqq. J. Magn. Magn. Mater. 138, 1994, pp. 255-269
 
\bibitem{back} Back, C. H., et al.  \flqq The 2020 Skyrmionics Roadmap \frqq. Journal of Physics D: Applied Physics, 2020

\bibitem{lacrox} Crépieux, A. et al. \flqq Dzyaloshinsky–Moriya interactions induced by symmetry breaking at a surface\frqq.  J. Magn. Magn. Mater. 182, 1998, pp. 341–349

\bibitem{fert} Fert, A. et al. \flqq Skyrmions on the track\frqq. Nature nanotechnology v. 8, 2013, pp. 152-155

\bibitem{garst_2017} Garst, M. et al. \flqq Collective spin excitations of helices and magnetic skyrmions: review and perspectives of magnonics in non-centrosymmetric magnets\frqq. Journal of Physics D: Applied Physics IOP Publishing 50, 2017, 293002

\bibitem{kitaev} Kitaev, A. \flqq Periodic table for topological insulators and superconductors\frqq. AIP Conference Proceedings 1134 (1), 2009, pp. 22-30

\bibitem{kiselevBogdanov} Kiselev, N. S. et al. \flqq Chiral skyrmions in thin magnetic films: new objects for magnetic storage technologies?\frqq. J. Phys. D v. 44, 2011, pp. 4

\bibitem{milde} Milde, P. et al. \flqq Unwinding of a Skyrmion Lattice by Magnetic Monopoles\frqq. Science Vol. 340, 2013, pp. 1076-1080

\bibitem{munzer} Munzer, W. et al. \flqq Emergent electrodynamics of skyrmions in a chiral magnet\frqq. Phys Rev B 81, 2010, pp. 301–304

\bibitem{rosch_muller} Müller J. et al, \flqq Capturing of a magnetic skyrmion with a hole \flqq. Phys. Rev. B 91, 2015, pp. 054410

\bibitem{mulhbauer} Mühlbauer, S. et al. \flqq Skyrmion Lattice in a Chiral Magnet\frqq. Science 323, 2009, pp. 915-919

\bibitem{Maleyev2006} Maleyev, S. V. \flqq Cubic magnets with Dzyaloshinskii-Moriya interaction at low temperature\frqq. Phys. Rev. B  73, 2006, 174402

\bibitem{nagaosa} Nagaosa, N. et al. \flqq Topological properties and dynamics of magnetic skyrmions\frqq. Nature nanotechnology 8, 2013, pp. 899-909

\bibitem{rosch_pfleiderer} Pfleiderer, C. et al. \flqq Single skyrmions spotted\frqq. Nature 465, 2010, pp. 880–881

\bibitem{rajaraman} Rajaraman, R. \flqq Solitons and instantons\frqq. Amsterdam, Netherlands: North-holland, 1982

\bibitem{roslerBogdanov} Roessler, U. K. et al. \flqq Spontaneous skyrmion ground
states in magnetic metals\frqq. Nature 442, 2006, pp. 797–801

\bibitem{Rybakov2015} Rybakov, F. N. et al. \flqq New spiral state and skyrmion lattice in 3D model of chiral magnets\frqq. New J. Phys. 18, 2016, 045002

\bibitem{roldan} Roldán-Molina, A. S. Nunez et al. \flqq Topological spin waves in the atomic-scale magnetic skyrmion crystal\frqq. New J. Phys. 18, 2016, 045015

\bibitem{skyrme} Skyrme, T. \flqq A unified field theory of mesons and baryons\frqq. Nuclear Physics. 31, 1962, pp. 556–569.

\bibitem{garst} Sch{\"u}tte, C. et. al. \flqq Magnon-skyrmion scattering in chiral magnets\frqq. Phys. Rev. B 90, 2014, 094423 

\bibitem{nagaosaHan} Yu, X. Z. et al. \flqq Skyrmion lattice in a two-dimensional chiral magnet\frqq. Phys. Rev. B 82, 2010, 094429

\bibitem{yu} Yu, X. Z. et al. \flqq Real-space observation of a two-dimensional skyrmion crystal\frqq. Nature 465, 2010, pp. 901-904

\bibitem{zhang} Zhang, X. et al. \flqq Magnetic skyrmion transistor: skyrmion motion in a voltage-gated nanotrack\frqq. Sci. Rep. 5, 2015, 11369

\end{thebibliography}
\end{document}