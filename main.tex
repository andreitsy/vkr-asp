% ОБЯЗАТЕЛЬНО ИМЕННО ТАКОЙ documentclass!
% (Основной кегль = 14pt, поэтому необходим extsizes)
% Формат, разумеется, А4
% article потому что стандарт не подразумевает разделов
% Глава = section, Параграф = subsection
% (понятия "глава" и "параграф" из стандарта)
\documentclass[a4paper,article,14pt]{extarticle}

% Подключаем главный пакет со всем необходимым
\usepackage{spbudiploma}

% Пакеты по желанию (самые распространенные)
% Хитрые мат. символы
\usepackage{euscript}
% Таблицы
\usepackage{longtable}
\usepackage{makecell}
% Картинки
\usepackage[pdftex]{graphicx}

\usepackage{amsthm,amssymb, amsmath}
\usepackage{textcomp}


\begin{document}

% Титульник в файле titlepage.tex
\begin{titlepage}
\newpage

\begin{center}
САНКТ-ПЕТЕРБУРГСКИЙ ГОСУДАРСТВЕННЫЙ УНИВЕРСИТЕТ\\
\vspace{1cm}

%\hrulefill
\end{center}

% \begin{flushright}
% На правах рукописи
% \end{flushright}
% \begin{flushright}
% \includegraphics[width=0.27\linewidth]{pic/signature.pdf}
% \end{flushright}

\vspace{0.5cm}
\begin{center}
ЦЫПИЛЬНИКОВ Андрей Васильевич
\end{center}

\vspace{1cm}
\begin{center}
    \textbf{выпускная квалификационная работа}
\end{center}{}
\vspace{1cm}

\begin{center}
\Large{\bf Спиновые волны в скирмионном кристалле}
\end{center}
\vspace{1cm}
\begin{center}
Направление 03.06.01 «Физика и астрономия» \\
Основная образовательная программа MK.3008.2016 «Физика»
\end{center}
\vspace{1cm}


\begin{flushleft}
\hspace{\stretch{1}} Научный руководитель:\\
\hspace{\stretch{1}} Аристов Дмитрий Николаевич\\
\hspace{\stretch{1}} д. ф.-м. н., проф.\\
\hspace{\stretch{1}} Рецензент:\\
\hspace{\stretch{1}} Демидов Юрий Андреевич\\
\hspace{\stretch{1}} к. ф.-м. н.\\
\vspace{1.5em}
\end{flushleft}

\vspace{\fill}

\begin{center}
Санкт-Петербург -- 2020
\end{center}
\end{titlepage}


% Содержание
\tableofcontents
\pagebreak

\specialsection{Введение}


\specialsection{Постановка задачи}


\specialsection{Обзор литературы}


\section{Пред. введение}
\subsection{Мотивация}


\subsection{Постановка задачи}


\subsection{Доступные программные средства}




\subsection{Полученные результаты} 


\section{Основная часть раз}


\pagebreak
\section{Основная часть два: Теория}

\section{Основная часть два: Детали реализации}
\subsection{Расчётная часть}

\section{Анализ экспериментов.}


\specialsection{Выводы}


\pagebreak

\specialsection{Заключение}


% Библиография в cpsconf стиле
% Аргумент {1} ниже включает переопределенный стиль с выравниванием слева
\begin{thebibliography}{1}
\bibitem{voc} Griffin D.W., Lim J.S. \flqq Multiband excitation vocoder\frqq. IEEE ASSP-36 (8), 1988, pp. 1223-1235.
\end{thebibliography}
\end{document}